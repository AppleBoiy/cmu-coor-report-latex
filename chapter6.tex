\chapter{สรุปผลการศึกษาและวิจารณ์ผลการศึกษา}\label{ch:conclusion}

\paragraph{}
บทนี้เป็นการสรุปภาพรวมทั้งหมดของโครงการวิจัย KGs-Augmented Testsuite Generator Framework ตั้งแต่ผลการดำเนินงานที่สำเร็จลุล่วงตามวัตถุประสงค์ ข้อสังเกตและข้อวิจารณ์ที่ได้จากการดำเนินงาน ข้อเสนอแนะสำหรับการพัฒนาต่อยอดในอนาคต ไปจนถึงประสบการณ์ที่ได้รับจากการเข้าร่วมโครงการสหกิจศึกษา

\section{สรุปผลการดำเนินงาน}
\paragraph{}
โครงการวิจัยนี้ประสบความสำเร็จในการพัฒนา KGs-Augmented Testsuite Generator Framework ซึ่งเป็นกรอบการทำงานที่สามารถแก้ไขปัญหาประสิทธิภาพในการค้นพบกรณีขอบเขต (Edge-Case Discovery Efficiency) สำหรับการทดสอบระบบขับขี่อัตโนมัติ (ADS) ได้อย่างเป็นรูปธรรม โดยกรอบการทำงานได้บูรณาการเทคโนโลยีสามส่วนหลักเข้าด้วยกัน คือ 1) Schema-guided LLM สำหรับการสกัดข้อมูลที่มีโครงสร้างจากรายงานอุบัติเหตุจริง 2) Knowledge Graph (KG) เพื่อทำหน้าที่เป็นแกนหลักเชิงความหมายที่ช่วยรักษาความต่อเนื่องเชิงเหตุผลและอนุมานข้อมูลได้ และ 3) การใช้ ODD Modular ร่วมกับเทคนิค Query Rotation เพื่อมุ่งเป้าการค้นหาไปยังสถานการณ์ที่ท้าทายระบบโดยตรง ผลลัพธ์สุดท้ายของกรอบการทำงานคือ ชุดพารามิเตอร์ (Parameter Set) คุณภาพสูง ที่พร้อมสำหรับนำไปใช้กับ Scenario Template มาตรฐานอุตสาหกรรม (ASAM OpenSCENARIO) ซึ่งถือเป็นการบรรลุวัตถุประสงค์ของโครงการที่ต้องการสร้างกระบวนการทดสอบที่เป็นระบบและมีประสิทธิภาพได้อย่างครบถ้วน

\section{ข้อสังเกตและข้อวิจารณ์}
\paragraph{}
จากการดำเนินโครงการ พบข้อสังเกตและประเด็นที่น่าสนใจหลายประการ ประการแรก การใช้ Knowledge Graph เป็นแกนหลักของข้อมูล พิสูจน์ให้เห็นว่ามีประสิทธิภาพสูงกว่าการจัดเก็บข้อมูลแบบตารางทั่วไปอย่างมีนัยสำคัญ เนื่องจากโครงสร้างกราฟเอื้อให้สามารถทำการสืบค้นเชิงความสัมพันธ์ที่ซับซ้อนและทำการอนุมาน (Inference) เพื่อค้นหา Edge-Case ที่ละเมิดกฎ ODD ได้โดยตรง ซึ่งเป็นหัวใจสำคัญของกรอบการทำงานนี้

\paragraph{}
ประการที่สอง แม้ว่า LLM จะเป็นเครื่องมือที่ทรงพลังในการสกัดข้อมูล แต่ก็ยังต้องการ "นั่งร้าน" (Scaffolding) ที่แข็งแรง กล่าวคือต้องอาศัยการออกแบบ Schema ที่ดีเยี่ยมและเทคนิค Multi-Step Prompting เพื่อควบคุมให้ผลลัพธ์มีความน่าเชื่อถือและลดปัญหา Hallucination ซึ่งเป็นการตอกย้ำว่าความสำเร็จของโครงการไม่ได้ขึ้นอยู่กับตัวโมเดลภาษาเพียงอย่างเดียว แต่ขึ้นอยู่กับสถาปัตยกรรมของกรอบการทำงานทั้งหมดที่ออกแบบมาเพื่อควบคุมและใช้งาน LLM อย่างมีเป้าหมาย

\paragraph{}
อย่างไรก็ตาม กรอบการทำงานยังมีจุดที่ควรปรับปรุง จากผลการประเมิน Scenario Completeness Score (R-score) พบว่าระบบสามารถสร้างสถานการณ์ที่มีปฏิสัมพันธ์ชัดเจน (เช่น การชนท้าย) ได้อย่างสมจริง แต่ยังคงมีความท้าทายในการสร้างสถานการณ์ที่ขึ้นอยู่กับปัจจัยที่กำกวมในรายงาน (เช่น สิ่งกีดขวางบนถนน) ซึ่งชี้ให้เห็นว่า คุณภาพของผลลัพธ์ยังคงขึ้นอยู่กับคุณภาพและความละเอียดของรายงานอุบัติเหตุต้นฉบับ เป็นอย่างมาก

\section{ข้อเสนอแนะในการพัฒนาต่อไป}
\paragraph{}
เพื่อให้กรอบการทำงานนี้มีศักยภาพสูงขึ้นในอนาคต มีแนวทางการพัฒนาต่อยอดที่น่าสนใจดังนี้:
\begin{enumerate}
    \item \textbf{การพัฒนาระบบตรวจสอบ KG อัตโนมัติ (Automated KG Validation):} พัฒนากลไกที่สามารถตรวจสอบความถูกต้องของข้อมูลใน Knowledge Graph ได้แบบกึ่งอัตโนมัติ เช่น การตรวจสอบความสอดคล้องกับกฎทางฟิสิกส์เบื้องต้น เพื่อลดภาระการตรวจสอบโดยผู้เชี่ยวชาญและเพิ่มความน่าเชื่อถือของข้อมูล
    \item \textbf{การขยายคลัง ODD Modular และการค้นพบอัตโนมัติ:} ขยายคลังของ ODD Modular ให้ครอบคลุมประเภทของ Edge-Case ที่มีความเฉพาะเจาะจงมากยิ่งขึ้น และอาจนำเทคนิค Machine Learning มาใช้เพื่อวิเคราะห์ KG และค้นหารูปแบบความเสี่ยงใหม่ๆ ที่สามารถนำมาสร้างเป็น ODD Modular ได้โดยอัตโนมัติ
    \item \textbf{การบูรณาการแบบครบวงจร (End-to-End Integration):} พัฒนาระบบให้ครบวงจร โดยเชื่อมต่อผลลัพธ์ (Parameter Set) เข้ากับระบบจำลองสถานการณ์ (เช่น CARLA) โดยตรง เพื่อให้สามารถสร้างและรัน Test Case ได้โดยอัตโนมัติ พร้อมทั้งเก็บผลการทดสอบและนำมาวิเคราะห์ได้ทันที
    \item \textbf{การจำลองปัจจัยมนุษย์ (Human Factors):} เพิ่มความสามารถในการวิเคราะห์และจำลองปัจจัยมนุษย์ในระดับที่ละเอียดขึ้น เช่น การจำลองความเหนื่อยล้า, การเสียสมาธิ, หรือเวลาในการตอบสนอง (Reaction Time) ของผู้ขับขี่เข้าไปใน Ontology ของ KG เพื่อสร้างสถานการณ์ที่สมจริงยิ่งขึ้น
\end{enumerate}

\section{ประสบการณ์จากการเข้าร่วมโครงการสหกิจศึกษา}
\paragraph{}
การเข้าร่วมโครงการสหกิจศึกษา ณ Japan Advanced Institute of Science and Technology (JAIST) เป็นประสบการณ์อันล้ำค่าที่มอบความรู้และทักษะที่สำคัญอย่างยิ่ง ทั้งในด้านเทคนิค การวิจัย และการทำงานในระดับสากล ข้าพเจ้าได้เรียนรู้และลงมือปฏิบัติจริงกับเทคโนโลยีขั้นสูงอย่าง Knowledge Graphs และ Large Language Models ได้ฝึกฝนการออกแบบและพัฒนากรอบการทำงานวิจัยตั้งแต่การทบทวนวรรณกรรม การระบุปัญหา การออกแบบขั้นตอนวิธี ไปจนถึงการประเมินผลและเขียนรายงานฉบับสมบูรณ์ การได้ทำงานร่วมกับ Professor Toshiaki Aoki และ Associate Professor Natthawut Kertkeidkachorn ทำให้ข้าพเจ้าได้เรียนรู้กระบวนการแก้ปัญหาที่เป็นระบบ การคิดเชิงวิพากษ์ และได้รับคำแนะนำที่เป็นประโยชน์อย่างยิ่ง ประสบการณ์ทั้งหมดนี้ไม่เพียงแต่พัฒนาทักษะทางวิชาชีพ แต่ยังสร้างแรงบันดาลใจและเป็นรากฐานที่แข็งแกร่งสำหรับการทำงานหรือศึกษาต่อในสายงานด้านปัญญาประดิษฐ์และระบบอัตโนมัติในอนาคต