\chapter{บทนำ}\label{ch:introduction}

\paragraph{}{\sloppy การปฏิบัติงานสหกิจครั้งนี้ผู้จัดทำได้ปฏิบัติงานที่ Japan Advanced Instituted of Science and Technology (JAIST) ซึ่งได้รับมอบหมายงานเกี่ยวกับการออกแบบเฟรมเวิร์ค (Framework) สำหรับการสร้างชุดทดสอบของระบบขับเคลื่อนอัตโนมัติของ รถยนต์โดยนำเอาความรู้เรื่องโครงสร้างกราฟความรู้ (Knowledge Graph) มาประยุกต์ใช้ \par}

\section{ข้อมูลสถานประกอบการ}\label{sec:company-info}

\subsection{ชื่อองค์กร}\label{subsec:company-name}
\paragraph{}Japan Advanced Instituted of Science and Technology (JAIST)

\subsection{ระยะเวลาการปฏิบัติงาน}\label{subsec:duration}
\paragraph{}ตั้งแต่วันที่ 14 เมษายน พ.ศ.2568 จนถึงวันที่ 30 กันยายน พ.ศ.2568

\subsection{ลักษณะขององค์กร}\label{subsec:company-type}
\paragraph{}Japan Advanced Institute of Science and Technology เป็นมหาวิทยาลัยในประเทศญี่ปุ่นที่จัดการศึกษาในระดับบัณฑิตศึกษา ซึ่งแบ่งสาขาวิชาตามหัวข้อศึกษาหลักที่มีอยู่ 3 หัวข้อดังนี้

\begin{enumerate}[label=\arabic*.)]
    \item Knowledge Science: สาขาวิชาที่บูรณาการความรู้เกี่ยวกับวิธีการออกแบบ
การจัดการธุรกิจ วิทยาศาสตร์ระบบ (System Science)
และความรู้อื่นๆ ที่เกี่ยวข้องกับปัญหาของมนุษย์ องค์กร
หรือสังคมเพื่อเสนอวิธีแก้ปัญหาเหล่านั้นและพิจารณาว่าจะทำให้วิธีแก้ปัญหาเป็นรูปธรรมได้อย่างไร
    \item Information Science: เป็นสาขาวิชาที่มุ่งเน้นการแก้ไขปัญหาสำหรับมนุษย์และสังคม
การสร้างทฤษฎีพื้นฐานใหม่ๆ ที่เป็นนวัตกรรมและการประยุกต์การประมวลผลข้อมูลเข้ากับการสื่อสารเพื่อรองรับสังคมยุคใหม่ที่ขับเคลื่อนด้วยข้อมูล
    \item Material Science: เป็นสาขาวิชาการที่มุ่งผลิตวัสดุใหม่และนวัตกรรมโดยมุ่งแก้ปัญหาให้กับมนุษยชาติและสังคม
และบุกเบิกสาขาที่ยังไม่มีการสำรวจบนพื้นฐานของฟิสิกส์ เคมี ชีววิทยา
และวิทยาศาสตร์และเทคโนโลยีที่เกี่ยวข้อง
\end{enumerate}

\section{ตำแหน่งและลักษณะงานที่ได้รับมอบหมาย}\label{sec:job-details}

\subsection{ตำแหน่งงานที่ปฏิบัติ}\label{subsec:job-position}
\paragraph{}Research Intern
\subsection{งานที่ได้รับมอบหมาย}\label{subsec:assigned-tasks}

\paragraph{}ออกแบบเฟรมเวิร์ค
(Framework)
สำหรับการสร้างชุดทดสอบประสิทธิภาพการทำงานของระบบ
ขับเคลื่อนอัตโนมัติของรถยนต์
เพื่อให้ผู้พัฒนาระบบขับเคลื่อนอัตโนมัติสามารถสร้างชุดทดสอบของตนเอง เพื่อนำไป
ปรับปรุง ปรับใช้ และทดสอบระบบของตนเอง

\section{หลักการและเหตุผล}\label{sec:introduction}


\paragraph{}งานวิจัย KGs-Augmented Test Suite Generation via Re-construct Accident Report มุ่งเน้นการแก้ปัญหาหลักในการทดสอบระบบขับขี่อัตโนมัติ (ADS) นั่นคือ ต้องสร้างเคสจำลองจำนวนมากจนกว่าจะพบ Edge-Case ใหม่ ปัจจุบัน การใช้ Large Language Models (LLMs) เพื่อสร้างสถานการณ์จำลองจากรายงานอุบัติเหตุจริงยังคงมีข้อจำกัดด้านความน่าเชื่อถือ เนื่องจากขาดโครงสร้างข้อมูลที่ชัดเจนและเป้าหมายการทดสอบที่สอดคล้องกับขอบเขตการทำงานของระบบ (ADS) ที่กำหนดไว้ ซึ่งส่งผลให้มีการสร้าง Scenario ที่ไม่จำเป็นจำนวนมาก จนกว่าจะพบเคสที่ท้าทายระบบจริง งานวิจัยนี้จึงถือกำเนิดขึ้นเพื่อพัฒนาแนวทางที่เป็นระบบ โดยมีวัตถุประสงค์หลักเพื่อ ลดจำนวนเหตุการณ์ที่ต้องสร้างใหม่ ให้ได้มากที่สุด และเพิ่มอัตราส่วนการค้นพบ Edge-Case

กรอบการทำงานที่นำเสนอประกอบด้วยการทำงานร่วมกันของเทคโนโลยีหลักสามส่วนเพื่อเพิ่มประสิทธิภาพดังกล่าว: 1) Schema-guided LLM ถูกใช้เพื่อสกัดข้อมูลอุบัติเหตุจากรายงานข้อความให้อยู่ในรูปแบบที่มีโครงสร้าง 2) Knowledge Graph (KG) ถูกใช้เป็นโครงสร้างเชิงความหมาย (Semantic Backbone) ในการจัดเก็บข้อมูล ทำให้สามารถอนุมานข้อมูลที่ขาดหายไปและรักษาความต่อเนื่องเชิงเหตุผลของเหตุการณ์ได้อย่างแม่นยำ และที่สำคัญที่สุดคือ 3) การผสานรวม Operational Design Domain (ODD) ที่กำหนดโดย JAMA เข้าไปใน KG โดย ODD นี้ทำหน้าที่เป็นไกด์ไลน์และตัวกรอง เพื่อจำกัดการสร้าง Scenario ให้มุ่งเน้นเฉพาะสถานการณ์ที่อยู่ในขอบเขตการทำงานของ ADS เท่านั้น ซึ่งถือเป็นการควบคุมทิศทางของการสร้างชุดทดสอบให้ มุ่งเป้าหมายสู่ Edge-Case ที่เกี่ยวข้อง โดยตรง

ผลลัพธ์ที่คาดว่าจะได้รับจากโครงการนี้คือ กรอบการทำงานที่เชื่อถือได้และมีประสิทธิภาพสูง ในการสร้างชุดทดสอบสำหรับ ADS ประโยชน์ที่สำคัญที่สุดคือการช่วยให้นักวิจัยและผู้พัฒนาสามารถ เพิ่มความเร็วในการค้นพบสถานการณ์ทดสอบที่สำคัญ ด้วยทรัพยากรที่น้อยลง ชุดทดสอบที่สร้างขึ้นจะมีความน่าเชื่อถือสูง เนื่องจากมีโครงสร้างและมีความสอดคล้องกับเงื่อนไข ODD ทำให้การประเมินความปลอดภัยของระบบ ADS มีความเข้มงวดและมีคุณภาพมากขึ้น ซึ่งเป็นการวางรากฐานสำคัญสำหรับการรับรองความปลอดภัยของยานยนต์อัตโนมัติในอนาคต

\section{วัตถุประสงค์}\label{sec:objectives}

\begin{enumerate}[label=\arabic*.)]
    \item พัฒนากรอบการสร้างชุดทดสอบที่มีความน่าเชื่อถือที่เรียกว่า KGs-Augmented Testsuite Generator ซึ่งสามารถสกัดข้อมูลอุบัติเหตุแบบมีโครงสร้างจากรายงานที่เป็นข้อความโดยใช้ Schema-guided LLM และสร้างเป็น Knowledge Graph (KG) เพื่อใช้เป็นรากฐานเชิงความหมายของข้อมูลอุบัติเหตุ
    \item เพิ่มประสิทธิภาพการค้นพบ Edge-Case โดยการบูรณาการ Operational Design Domain (ODD) ที่กำหนดโดย JAMA เข้ากับ Knowledge Graph เพื่อใช้เป็นตัวกรองและหลักเกณฑ์ในการควบคุมการสร้าง Scenario ให้มุ่งเน้นเฉพาะสถานการณ์ที่ท้าทายระบบ (Edge-Case) ซึ่งเป็นไปตามวัตถุประสงค์หลักของงานวิจัยคือ การลดจำนวนเหตุการณ์ที่จำเป็นต้องสร้างใหม่จนกว่าจะพบเเคสใหม่
    \item สร้างชุดทดสอบอุบัติเหตุที่มีโครงสร้างที่สมบูรณ์และถูกต้องตามหลักเหตุผล ซึ่งสามารถส่งออกในรูปแบบมาตรฐาน เช่น ASAM OpenSCENARIO เพื่อนำไปใช้งานในการจำลองการทดสอบ ระบบขับขี่อัตโนมัติได้อย่างมีประสิทธิภาพและตรงเป้าหมาย
\end{enumerate}

\section{ประโยชน์ที่คาดว่าจะได้รับ}\label{sec:expected-benefits}

\begin{enumerate}[label=\arabic*.)]
    \item การเพิ่มประสิทธิภาพในการค้นพบ Edge-Case: กรอบการทำงานนี้จะช่วยให้นักวิจัยและผู้พัฒนาระบบสามารถลดจำนวนเหตุการณ์จำลองที่ไม่จำเป็นลงได้อย่างมาก เนื่องจากมีการใช้ ODD เป็นไกด์ไลน์ในการกรองและควบคุมการสร้าง Scenario ให้มุ่งเน้นเฉพาะสถานการณ์ที่อยู่ในขอบเขตการทำงานของ ADS และท้าทายระบบจริง ๆ เท่านั้น ซึ่งนำไปสู่การประหยัดเวลาและทรัพยากรในการทดสอบ
    \item การยกระดับความน่าเชื่อถือของชุดทดสอบ: ชุดทดสอบที่สร้างขึ้นจาก Knowledge Graph มีความถูกต้องเชิงโครงสร้างและรักษาความต่อเนื่องเชิงเหตุผล (Causal Continuity) ของเหตุการณ์อุบัติเหตุ ทำให้ผลการประเมินระบบขับขี่อัตโนมัติมีความน่าเชื่อถือและสอดคล้องกับความเป็นจริงมากขึ้น
    \item การเป็นรากฐานสำหรับงานวิจัยต่อยอด: Knowledge Graph ที่สร้างขึ้นจากข้อมูลอุบัติเหตุจริงและผสานรวมกับเงื่อนไข ODD สามารถทำหน้าที่เป็นแหล่งข้อมูลความรู้เชิงความหมายที่มีโครงสร้าง ซึ่งเป็นประโยชน์อย่างยิ่งในการพัฒนากฎความปลอดภัย (Safety Rules) การสร้าง Ontology สำหรับการขับขี่อัตโนมัติ หรือการพัฒนาเครื่องมือประเมินความเสี่ยงอื่น ๆ ในอนาคต
    \item การสนับสนุนการรับรองความปลอดภัยตามมาตรฐาน: กรอบการทำงานนี้ช่วยให้มั่นใจได้ว่า Scenario ที่ใช้ในการทดสอบมีความสอดคล้องกับเงื่อนไขการปฏิบัติงานที่กำหนด (ODD) ของ JAMA ซึ่งเป็นปัจจัยสำคัญในการดำเนินการและสนับสนุนกระบวนการขอการรับรองความปลอดภัยของยานยนต์อัตโนมัติ
\end{enumerate}

\section{ขอบเขต}\label{sec:scope}

\subsection{ขอบเขตของข้อมูล}\label{subsec:data-scope}
\paragraph{}รายงานจาก CIREN (Crash Injury Research and Engineering Network) ของสหรัฐอเมริกา และรายงานจาก GIDAS (German In-Depth Accident Study) ข้อมูลนำเข้าเหล่านี้อยู่ในรูปแบบของรายงานอุบัติเหตุที่เป็นข้อความแบบไม่มีโครงสร้าง (Unstructured Text Reports) ซึ่งจำเป็นต้องผ่านกระบวนการสกัดข้อมูลที่มีโครงสร้าง (Structured Data Extraction) โดยใช้ Schema-guided LLM ก่อนนำไปสร้างเป็น Knowledge Graph ทั้งนี้ ข้อจำกัดด้านการใช้งานคือ ข้อมูลทั้งหมดจะถูกใช้เพื่อสกัดเอนทิตี ความสัมพันธ์ และเงื่อนไขที่จำเป็นสำหรับการสร้าง Knowledge Graph และ Scenario จำลองเท่านั้น โดยไม่รวมถึงข้อมูลส่วนบุคคลหรือข้อมูลระบุตัวตนอื่น ๆ ที่เกี่ยวข้องกับบุคคลในรายงาน

\subsection{ขอบเขตของงาน}\label{subsec:work-scope}

\paragraph{}ขอบเขตของงานวิจัยนี้ครอบคลุมกิจกรรมหลักตั้งแต่การประมวลผลข้อมูลอุบัติเหตุไปจนถึงการสร้างชุดทดสอบที่มีโครงสร้างที่มุ่งเน้นเป้าหมาย โดยสามารถสรุปขอบเขตของการดำเนินงานได้ดังนี้:

\begin{enumerate}[label=\arabic*.)]
    \item การพัฒนากรอบการทำงาน KGs-Augmented Testsuite Generator ซึ่งใช้ Schema-guided LLM ในการสกัดข้อมูล และใช้ Knowledge Graph (KG) ในการจัดเก็บข้อมูลเชิงความหมายพร้อมรองรับ Inference Engine
    \item การบูรณาการ ODD: นำ Operational Design Domain (ODD) ที่กำหนดโดย JAMA มาผสานรวมเข้ากับ Knowledge Graph เพื่อทำหน้าที่เป็นเงื่อนไขในการกรองและควบคุมการสร้าง Scenario ให้มุ่งเป้าหมายเฉพาะ Edge-Case ที่เกี่ยวข้องกับขอบเขตการทำงานของระบบ
    \item การสร้างผลลัพธ์: สร้างชุดข้อมูล Scenario อุบัติเหตุที่มีโครงสร้างสมบูรณ์ ซึ่งสามารถส่งออกในรูปแบบมาตรฐานของอุตสาหกรรม เช่น ASAM OpenSCENARIO ที่พร้อมนำไปใช้งานในสภาพแวดล้อมจำลอง
\end{enumerate}

\section{เครื่องมือและเทคโนโลยีที่ใช้}\label{sec:tools-and-tech}
\subsection{ฮาร์ดแวร์ที่ใช้ในการปฏิบัติงาน}\label{subsec:hardware-used}
\begin{itemize}
\item Operating System: Ubuntu 24.04.2 LTS
\item Processor: AMD Ryzen 7 5700G
\item Graphic card: Nvidia RTX 4000 Ada generation
\item Memory: 46GB
\item Storage: 1TB
\end{itemize}

\subsection{ซอฟต์แวร์ที่ใช้ในการปฏิบัติงาน}\label{subsec:software-used}
\begin{enumerate}[label=\arabic*.)]
    \item Protégé: ใช้ในการออกแบบ Schema ของ Knowledge Graph
    \item Carla: เป็นโปรแกรมจำลองสถานการณ์การขับขี่แบบ Open
    \item MATLAB: ใช้สำหรับงานวิเคราะห์ข้อมูลโครงสร้างและเครือข่ายของถนน
    \item Large Language Models (LLMs): ใช้ GPT-4o ใช้สำหรับงานประมวลผลภาษาธรรมชาติ เพื่อช่วยในการ สร้าง (generation) และจัดการข้อมูลสำหรับสถานการณ์จำลอง และการสร้าง Knowledge Graph ตามที่ ระบุในแผนภาพระบบ
\end{enumerate}

\subsection{ภาษาโปรแกรมที่ใช้ในการพัฒนา}\label{subsec:programming-languages}
\begin{enumerate}[label=\arabic*.)]
\item Python: ใช้สำหรับการพัฒนา Framework หลักในการสกัดข้อมูลอุบัติเหตุ การสร้าง Knowledge Graph และการผสานรวม ODD
\item Cypher: ใช้สำหรับการสืบค้นและจัดการข้อมูลในฐานข้อมูล Knowledge Graph
\item SQL: ใช้สำหรับการจัดการและสืบค้นข้อมูลในฐานข้อมูลเชิงสัมพันธ์ (Relational Database)
\item Shell Scripting: ใช้สำหรับการจัดการงานอัตโนมัติและการตั้งค่าสภาพแวดล้อมการพัฒนา
\item LaTeX: ใช้สำหรับการจัดทำรายงานและเอกสารทางวิชาการ
\item MATLAB: ใช้สำหรับการวิเคราะห์ข้อมูลโครงสร้างและเครือข่ายของถนน
\end{enumerate}

\section{แผนปฏิบัติงานสหกิจศึกษา}\label{sec:work-plan}

\begin{table}[htbp]
    \centering
    \caption{ตารางสรุปแผนการดำเนินงานวิจัย}
    \label{tab:project-timeline-revised}
\begin{tabular}{|c|p{7cm}|c|c|c|c|c|}
\hline
\multicolumn{1}{|c|}{ลำดับ} & \multicolumn{1}{|c|}{กิจกรรมหลัก} & \multicolumn{1}{|c|}{พ.ค.} & \multicolumn{1}{|c|}{มิ.ย.} & \multicolumn{1}{|c|}{ก.ค.} & \multicolumn{1}{|c|}{ส.ค.} & \multicolumn{1}{|c|}{ก.ย.}\\
\hline
1 & การเรียนรู้พื้นฐานและการทบทวนงานวิจัย (LLM, KG, ODD)
  & \cellcolor{gray!40} & \cellcolor{gray!40} &  &  &  \\
\hline
2 & การสรุปแผนวิจัยโดยละเอียดและการออกแบบ Schema/Ontology สำหรับอุบัติเหตุ
  &  & \cellcolor{gray!40} & \cellcolor{gray!40} &  &  \\
\hline
3 & การจัดหาและเตรียมชุดข้อมูลอุบัติเหตุ (CIREN/GIDAS) สำหรับการสกัด
  &  & \cellcolor{gray!40} & \cellcolor{gray!40} &  &  \\
\hline
4 & การพัฒนา Schema-guided LLM สำหรับการสกัดข้อมูลอุบัติเหตุที่มีโครงสร้าง
  &  &  & \cellcolor{gray!40} & \cellcolor{gray!40} &  \\
\hline
5 & การสร้าง Knowledge Graph (KG) และการพัฒนา Inference Engine (รวม ODD)
  &  &  &  & \cellcolor{gray!40} & \cellcolor{gray!40} \\
\hline
6 & การบูรณาการระบบทั้งหมด (LLM $\rightarrow$ KG $\rightarrow$ Inference) และการสร้างชุดสถานการณ์ทดสอบเบื้องต้น
  &  &  &  & \cellcolor{gray!40} & \cellcolor{gray!40} \\
\hline
7 & การทดสอบหลัก (Main Experiment) และการสร้าง Edge-Case Scenario จำนวนมาก
  &  &  &  &  & \cellcolor{gray!40} \\
\hline
8 & การประเมินผลสถานการณ์ที่สร้างขึ้นและการวิเคราะห์สรุปผล
  &  &  &  &  & \cellcolor{gray!40} \\
\hline
9 & การเขียนรายงานฉบับสมบูรณ์และการเตรียมการเพื่อเผยแพร่ผลงานวิจัย
  &  &  &  & \cellcolor{gray!40} & \cellcolor{gray!40} \\
\hline
\end{tabular}
\end{table}