\documentclass[a4paper,12pt]{report}
% XeTeX/tectonic build: use fontspec + polyglossia (do not load inputenc/babel)
\usepackage[table]{xcolor}
\usepackage{graphicx}
\usepackage{titlesec}
\usepackage{fancyhdr}
\usepackage{geometry}
\usepackage{setspace}
\usepackage{tocloft}
\usepackage{hyperref}
\usepackage{polyglossia}
\setdefaultlanguage{thai}
\setotherlanguage{english}
\usepackage{array}

\geometry{margin=1in}
\setlength{\parindent}{2em}
\setlength{\parskip}{0pt} 
\doublespacing

% Define chapter title format
\titleformat{\chapter}[display]
  {\normalfont\Large\bfseries\centering}{บทที่ \thechapter}{1ex}{\Large}

% Define section and subsection
% Section title: Large, bold, with spacing before/after
\titleformat{\section}
  {\large\bfseries}   % Font style
  {\thesection}       % Section number
  {1em}               % Space between number and title
  {}                  % Before-code

% Subsection title: Normal size
\titleformat{\subsection}
  {\normalsize\bfseries}
  {\thesubsection}
  {1em}
  {}

% Header and footer
\pagestyle{fancy}
\fancyhf{}
\fancyfoot[C]{\thepage}

% thai language support
\usepackage{fontspec}
\setmainfont[
  Path = ./fonts/,                  % Path to the fonts folder
  UprightFont = THSarabun,           % Regular (upright) font
  BoldFont = THSarabun-Bold,         % Bold font
  ItalicFont = THSarabun-Italic,     % Italic font
  BoldItalicFont = THSarabun-BoldItalic, % Bold-Italic font
  Scale = 1.5
]{THSarabun}
% Explicitly bind Thai and Latin/English fonts for polyglossia
\newfontfamily\thaifont[
  Path = ./fonts/,
  UprightFont = THSarabun,
  BoldFont = THSarabun-Bold,
  ItalicFont = THSarabun-Italic,
  BoldItalicFont = THSarabun-BoldItalic,
  Script=Thai
]{THSarabun}
\newfontfamily\englishfont[
  Path = ./fonts/,
  UprightFont = THSarabun,
  BoldFont = THSarabun-Bold,
  ItalicFont = THSarabun-Italic,
  BoldItalicFont = THSarabun-BoldItalic
]{THSarabun}
\newfontfamily\latinfont[
  Path = ./fonts/,
  UprightFont = THSarabun,
  BoldFont = THSarabun-Bold,
  ItalicFont = THSarabun-Italic,
  BoldItalicFont = THSarabun-BoldItalic
]{THSarabun}
\XeTeXlinebreaklocale "th"
\XeTeXlinebreakskip = 0pt plus 1pt

% For dotted underline
\usepackage{tikz}
\newcommand{\uloosdot}[1]{%
    \tikz[baseline=(todotted.base)]{
        \node[inner sep=1pt,outer sep=0pt] (todotted) {#1};
        \draw[loosely dotted] (todotted.south west) -- (todotted.south east);
    }%
}%

% Thai page number
% Thai letters as page numbers
\newcommand{\thaipagenum}{
  \ifcase\value{page}
    \or ก%
    \or ข%
    \or ค%
    \or ง%
    \or จ%
    \or ฉ%
    \or ช%
    \or ซ%
    \or ฌ%
    \else หน้า~\thepage % fallback
  \fi
}

\usepackage{tocbibind}
\usepackage{indentfirst}
\emergencystretch=3em

\begin{document}

% Cover Page
\begin{titlepage}
    \thispagestyle{empty}
    \vspace*{\fill}  % Push content to vertical center
    \begin{center}
        \textbf{\Large รายงานปฏิบัติงานสหกิจศึกษา}\\[0.5cm]
        \textbf{\Large Japan Advanced Instituted of Science and Technology}\\[3cm]
        \Large \textbf{KGs-Augmented Test suite generation via Re-construct accident report}\\
        \textbf{\Large นาย ชัยภัทร ใจน่าน}\\
        \textbf{\Large 650510606}\\[3cm]
        \textbf{\large สหกิจศึกษานี้เป็นส่วนหนึ่งของการศึกษาหลักสูตรปริญญาวิทยาศาสตรบัณฑิต} \\
        \textbf{\large สาขาวิชาวิทยาการคอมพิวเตอร์}\\
        \textbf{\large คณะวิทยาศาสตร์ มหาวิทยาลัยเชียงใหม่}\\
        \textbf{\large ปีการศึกษา 2565}
    \end{center}
    \vspace*{\fill} 
\end{titlepage}

% Approval Page
\newpage
\clearpage
\thispagestyle{empty}
\setcounter{page}{0}

\begin{center}
    \large \textbf{รายงานปฏิบัติงานสหกิจศึกษา}\\[0.5cm]
    \textbf{(KGs-Augmented Test suite generation via Re-construct accident report)}\\[3cm]
    \normalsize นาย ชัยภัทร ใจน่าน\\
    650510606\\[3cm]
    สหกิจศึกษานี้เป็นส่วนหนึ่งของการศึกษาหลักสูตรปริญญาวิทยาศาสตรบัณฑิต \\
    สาขาวิชาวิทยาการคอมพิวเตอร์\\
    คณะวิทยาศาสตร์ มหาวิทยาลัยเชียงใหม่\\
    ปีการศึกษา 2565\\[2cm]
    \normalsize คณะกรรมการสอบสหกิจศึกษา\\[1cm]
    \begin{tabular}{c l}
        \uloosdot{\phantom{xxxxxxxxxxxxxxxxxxxxxxxxxxx}} & ประธานกรรมการ \\
        ( ผศ.ดร.อารีรัตน์ ตรงรัศมีทอง ) & \\
         & \\
        \uloosdot{\phantom{xxxxxxxxxxxxxxxxxxxxxxxxxxx}} & กรรมการ \\
        ( ผศ.ดร.เสมอแข สมหอม ) & \\
    \end{tabular} \\[1cm]
    วันที่ \uloosdot{\phantom{xxx}} เดือน \uloosdot{\phantom{xxxxxxxxx}} พ.ศ. \uloosdot{\phantom{xxxxxx}}
\end{center}

\newpage
\clearpage
\thispagestyle{empty}
\setcounter{page}{0}
หนังสือยิมยอมให้ข้อมูลเพื่อการศึกษา

\newpage
\clearpage
\renewcommand{\thepage}{\thaipagenum}
\setcounter{page}{1}

% Acknowledgements
\chapter*{กิตติกรรมประกาศ}
\addcontentsline{toc}{chapter}{กิตติกรรมประกาศ}
\paragraph{} ข้าพเจ้าขอขอบพระคุณอาจารย์ที่ปรึกษาสหกิจศึกษา ผศ.ดร.รัศมีทิพย์ วิตา ที่ได้ให้คำแนะนำ
และช่วยแก้ไขปัญหาต่าง ๆ ตลอดระยะเวลาการทำงาน
\begin{flushright}
    นาย ชัยภัทร ใจน่าน\\
    650510606
\end{flushright}
% (insert acknowledgements here)

% Abstract
{
\cleardoublepage% Move to first page of new chapter
\let\clearpage\relax% Don't allow page break
\noindent
\begin{tabular}{l p{10cm}}
    \textbf{หัวข้อสหกิจศึกษา} & การสร้างชุดทดสอบเสริมด้วย Knowledge Graphs ผ่านการสร้างรายงานอุบัติเหตุซ้ำ \vspace{0.2cm} \\
    \textbf{สถานประกอบการ} & Japan Advanced Instituted of Science and Technology \vspace{0.2cm}\\
    \textbf{ผู้ดำเนินการศึกษา} & ชัยภัทร ใจน่าน (Chaipat Jainan) \vspace{0.2cm}\\
    \textbf{หลักสูตร} & วิทยาศาสตรบัณฑิต สาขาวิทยาการคอมพิวเตอร์ \vspace{0.2cm}\\
    \textbf{อาจารย์ที่ปรึกษาสหกิจ} & ผู้ช่วยศาสตราจารย์ ดร.รัศมีทิพย์ วิตา \vspace{0.2cm}\\
\end{tabular}

\chapter*{บทคัดย่อ}
\addcontentsline{toc}{chapter}{บทคัดย่อ}

\paragraph{}
งานวิจัยนี้มีวัตถุประสงค์เพื่อแก้ไขปัญหาด้านความน่าเชื่อถือและประสิทธิภาพในการสร้างชุดทดสอบสำหรับระบบขับขี่อัตโนมัติ (ADS) จากรายงานอุบัติเหตุจริง แม้ว่าโมเดลภาษาขนาดใหญ่ (LLM) จะมีศักยภาพในการสร้างสถานการณ์จำลอง แต่การใช้งานโดยตรงยังขาดกลไกในการมุ่งเน้นไปยังกรณีขอบเขต (Edge-Case) ที่ชัดเจน ซึ่งส่งผลให้เกิดการสร้างสถานการณ์จำลองที่ไม่จำเป็นจำนวนมาก งานวิจัยจึงนำเสนอกรอบการทำงานสำหรับการสร้างชุดทดสอบที่น่าเชื่อถือและมุ่งเน้นเป้าหมาย โดยใช้ Schema-guided LLM ในการสกัดข้อมูลอุบัติเหตุที่มีโครงสร้าง ก่อนนำไปสร้างเป็น Knowledge Graph (KG) เพื่อใช้เป็นโครงสร้างเชิงความหมายที่ช่วยในการอนุมานข้อมูลและความต่อเนื่องเชิงเหตุผล หัวใจสำคัญของกรอบการทำงานนี้คือการบูรณาการ Operational Design Domain (ODD) ที่กำหนดโดย JAMA เข้าไปใน KG เพื่อใช้เป็นหลักเกณฑ์ในการกรองและควบคุมการสร้าง Scenario ให้มุ่งเน้นการค้นพบ Edge-Case ใหม่ ๆ ได้อย่างรวดเร็วและมีประสิทธิภาพ ผลลัพธ์ที่ได้คือก้าวสำคัญในการพัฒนาวิธีการสร้างชุดทดสอบที่มีความแม่นยำสูง ซึ่งช่วยลดจำนวนครั้งในการสร้างเหตุการณ์จำลองซ้ำ และสนับสนุนการรับรองความปลอดภัยของระบบ ADS ได้อย่างเป็นรูปธรรม
}


\newpage
{
\cleardoublepage% Move to first page of new chapter
\let\clearpage\relax% Don't allow page break
\noindent
\begin{tabular}{l l c}
    \textbf{Title} & KGs-Augmented Test suite generation via Re-construct accident report & \vspace{0.2cm}\\
    \textbf{Company} & Japan Advanced Instituted of Science and Technology & \vspace{0.2cm}\\
    \textbf{Name} & Chaipat Jainan & \vspace{0.2cm}\\
    \textbf{ID} & 650510606 & \vspace{0.2cm}\\
    \textbf{Degree} & Bachelor of Science in Computer Science & \vspace{0.2cm}\\
    \textbf{Advisor} & Asst. Prof. Ratsameetip Wita & \vspace{0.2cm}\\
\end{tabular}

\chapter*{Abstract}
\addcontentsline{toc}{chapter}{Abstract}

\paragraph{}
Ensuring the safety of Autonomous Driving Systems (ADS) necessitates rigorous testing using simulated scenarios derived from real-world accident reports. While Large Language Models (LLMs) offer a promising approach for generating these scenarios, their direct application in discovering elusive edge-cases is often inefficient and unreliable. This inefficiency stems primarily from the inconsistent nature of the source reports and the LLMs' tendency to generate numerous irrelevant scenarios, leading to an unnecessarily high number of reconstruction efforts before a novel test case is found.

To address this challenge, this study proposes a novel KGs-Augmented Testsuite Generator Framework designed to create reliable and goal-oriented test suites. The framework employs a schema-guided LL to accurately extract structured accident data from unstructured reports. This data is then modeled as a Knowledge Graph (KG), which serves as a semantic backbone, enabling robust inference of missing information and maintaining the causal continuity between events. Crucially, the framework integrates the Operational Design Domain (ODD), utilizing constraints defined by JAMA, directly into the KG structure.

The integration of ODD acts as a powerful filter and guideline, controlling the scenario generation process to focus exclusively on test cases that violate or push the boundaries of the defined operational limits. This targeted approach is the key to the framework's effectiveness, which aims to significantly reduce the number of reconstructed scenarios required to discover a new, challenging edge-case. The resulting structured scenarios are highly reliable and compliant with industry standards (e.g., ASAM OpenSCENARIO), marking a substantial improvement in the efficiency and quality of ADS safety evaluation.
}
\newpage
% Table of Contents
\renewcommand{\contentsname}{สารบัญ}
\renewcommand{\listfigurename}{สารบัญรูป}
\renewcommand{\listtablename}{สารบัญตาราง}
\tableofcontents
\newpage
\listoffigures
\newpage
\listoftables

% Revert back to arabic page number
\cleardoublepage
\renewcommand{\thepage}{\arabic{page}}
\setcounter{page}{1}
\pagenumbering{arabic}

% Chapters
\renewcommand{\figurename}{รูปที่}
\renewcommand{\tablename}{ตารางที่}
\renewcommand\bibname{บรรณานุกรม}
\chapter{บทนำ}\label{ch:introduction}

\paragraph{}{\sloppy การปฏิบัติงานสหกิจครั้งนี้ผู้จัดทำได้ปฏิบัติงานที่ Japan Advanced Instituted of Science and Technology (JAIST) ซึ่งได้รับมอบหมายงานเกี่ยวกับการออกแบบเฟรมเวิร์ค (Framework) สำหรับการสร้างชุดทดสอบของระบบขับเคลื่อนอัตโนมัติของ รถยนต์โดยนำเอาความรู้เรื่องโครงสร้างกราฟความรู้ (Knowledge Graph) มาประยุกต์ใช้ ข้อมูลสถานประกอบการ\par}

\section{ข้อมูลสถานประกอบการ}\label{sec:company-info}

\subsection{ชื่อองค์กร}\label{subsec:company-name}
Japan Advanced Instituted of Science and Technology (JAIST)

\subsection{ระยะเวลาการปฏิบัติงาน}\label{subsec:duration}
ตั้งแต่วันที่ 14 เมษายน พ.ศ.2568 จนถึงวันที่ 30 กันยายน พ.ศ.2568

\subsection{ลักษณะขององค์กร}\label{subsec:company-type}
Japan Advanced Institute of Science and Technology เป็นมหาวิทยาลัยในประเทศญี่ปุ่นที่จัดการศึกษาในระดับบัณฑิตศึกษา ซึ่งแบ่งสาขาวิชาตามหัวข้อศึกษาหลักที่มีอยู่ 3 หัวข้อดังนี้

\begin{enumerate}
    \item
Knowledge Science: สาขาวิชาที่บูรณาการความรู้เกี่ยวกับวิธีการออกแบบ
การจัดการธุรกิจ วิทยาศาสตร์ระบบ (System Science)
และความรู้อื่นๆ ที่เกี่ยวข้องกับปัญหาของมนุษย์ องค์กร
หรือสังคมเพื่อเสนอวิธีแก้ปัญหาเหล่านั้นและพิจารณาว่าจะทำให้วิธีแก้ปัญหาเป็นรูปธรรมได้อย่างไร
    \item
Information Science: เป็นสาขาวิชาที่มุ่งเน้นการแก้ไขปัญหาสำหรับมนุษย์และสังคม
การสร้างทฤษฎีพื้นฐานใหม่ๆ ที่เป็นนวัตกรรมและการประยุกต์การประมวลผลข้อมูลเข้ากับการสื่อสารเพื่อรองรับสังคมยุคใหม่ที่ขับเคลื่อนด้วยข้อมูล
    \item
Material Science: เป็นสาขาวิชาการที่มุ่งผลิตวัสดุใหม่และนวัตกรรมโดยมุ่งแก้ปัญหาให้กับมนุษยชาติและสังคม
และบุกเบิกสาขาที่ยังไม่มีการสำรวจบนพื้นฐานของฟิสิกส์ เคมี ชีววิทยา
และวิทยาศาสตร์และเทคโนโลยีที่เกี่ยวข้อง
\end{enumerate}

\section{ตำแหน่งและลักษณะงานที่ได้รับมอบหมาย}\label{sec:job-details}

\subsection{ตำแหน่งงานที่ปฏิบัติ}\label{subsec:job-position}
Research Intern
\subsection{งานที่ได้รับมอบหมาย}\label{subsec:assigned-tasks}

\paragraph{}ออกแบบเฟรมเวิร์ค
(Framework)
สำหรับการสร้างชุดทดสอบประสิทธิภาพการทำงานของระบบ
ขับเคลื่อนอัตโนมัติของรถยนต์
เพื่อให้ผู้พัฒนาระบบขับเคลื่อนอัตโนมัติสามารถสร้างชุดทดสอบของตนเอง เพื่อนำไป
ปรับปรุง ปรับใช้ และทดสอบระบบของตนเอง

\section{หลักการและเหตุผล}\label{sec:introduction}


\paragraph{}งานวิจัย KGs-Augmented Test suite generation via Re-construct accident report มุ่งเน้นการแก้ปัญหาหลักในการทดสอบระบบขับขี่อัตโนมัติ (ADS) นั่นคือ ความไร้ประสิทธิภาพในการค้นพบ Edge-Case ใหม่ ๆ อย่างรวดเร็ว ปัจจุบัน การใช้ Large Language Models (LLMs) เพื่อสร้างสถานการณ์จำลองจากรายงานอุบัติเหตุจริงยังคงมีข้อจำกัดด้านความน่าเชื่อถือ เนื่องจากขาดโครงสร้างข้อมูลที่ชัดเจนและเป้าหมายการทดสอบที่สอดคล้องกับขอบเขตการทำงานของระบบ (ADS) ที่กำหนดไว้ ซึ่งส่งผลให้มีการสร้าง Scenario ที่ไม่จำเป็นจำนวนมาก จนกว่าจะพบเคสที่ท้าทายระบบจริง งานวิจัยนี้จึงถือกำเนิดขึ้นเพื่อพัฒนาแนวทางที่เป็นระบบ โดยมีวัตถุประสงค์หลักเพื่อ ลดจำนวนเหตุการณ์ที่ต้องสร้างใหม่ ให้ได้มากที่สุด และเพิ่มอัตราส่วนการค้นพบ Edge-Case

กรอบการทำงานที่นำเสนอประกอบด้วยการทำงานร่วมกันของเทคโนโลยีหลักสามส่วนเพื่อเพิ่มประสิทธิภาพดังกล่าว: 1) Schema-guided LLM ถูกใช้เพื่อสกัดข้อมูลอุบัติเหตุจากรายงานข้อความให้อยู่ในรูปแบบที่มีโครงสร้าง 2) Knowledge Graph (KG) ถูกใช้เป็นโครงสร้างเชิงความหมาย (Semantic Backbone) ในการจัดเก็บข้อมูล ทำให้สามารถอนุมานข้อมูลที่ขาดหายไปและรักษาความต่อเนื่องเชิงเหตุผลของเหตุการณ์ได้อย่างแม่นยำ และที่สำคัญที่สุดคือ 3) การผสานรวม Operational Design Domain (ODD) ที่กำหนดโดย JAMA เข้าไปใน KG โดย ODD นี้ทำหน้าที่เป็นไกด์ไลน์และตัวกรอง เพื่อจำกัดการสร้าง Scenario ให้มุ่งเน้นเฉพาะสถานการณ์ที่อยู่ในขอบเขตการทำงานของ ADS เท่านั้น ซึ่งถือเป็นการควบคุมทิศทางของการสร้างชุดทดสอบให้ มุ่งเป้าหมายสู่ Edge-Case ที่เกี่ยวข้อง โดยตรง

ผลลัพธ์ที่คาดว่าจะได้รับจากโครงการนี้คือ กรอบการทำงานที่เชื่อถือได้และมีประสิทธิภาพสูง ในการสร้างชุดทดสอบสำหรับ ADS ประโยชน์ที่สำคัญที่สุดคือการช่วยให้นักวิจัยและผู้พัฒนาสามารถ เพิ่มความเร็วในการค้นพบสถานการณ์ทดสอบที่สำคัญ ด้วยทรัพยากรที่น้อยลง ชุดทดสอบที่สร้างขึ้นจะมีความน่าเชื่อถือสูง เนื่องจากมีโครงสร้างและมีความสอดคล้องกับเงื่อนไข ODD ทำให้การประเมินความปลอดภัยของระบบ ADS มีความเข้มงวดและมีคุณภาพมากขึ้น ซึ่งเป็นการวางรากฐานสำคัญสำหรับการรับรองความปลอดภัยของยานยนต์อัตโนมัติในอนาคต

\section{วัตถุประสงค์}\label{sec:objectives}

\begin{enumerate}
    \item พัฒนากรอบการสร้างชุดทดสอบที่มีความน่าเชื่อถือ: เพื่อพัฒนากรอบการทำงานที่เรียกว่า KGs-Augmented Testsuite Generator ซึ่งสามารถสกัดข้อมูลอุบัติเหตุแบบมีโครงสร้างจากรายงานที่เป็นข้อความโดยใช้ Schema-guided LLM และสร้างเป็น Knowledge Graph (KG) เพื่อใช้เป็นรากฐานเชิงความหมายของข้อมูลอุบัติเหตุ
    \item เพิ่มประสิทธิภาพการค้นพบ Edge-Case: เพื่อบูรณาการ Operational Design Domain (ODD) ที่กำหนดโดย JAMA เข้ากับ Knowledge Graph เพื่อใช้เป็นตัวกรองและหลักเกณฑ์ในการควบคุมการสร้าง Scenario ให้มุ่งเน้นเฉพาะสถานการณ์ที่ท้าทายระบบ (Edge-Case) ซึ่งเป็นไปตามวัตถุประสงค์หลักของงานวิจัยคือ การลดจำนวนเหตุการณ์ที่จำเป็นต้องสร้างใหม่จนกว่าจะพบเเคสใหม่
    \item สร้างชุดทดสอบที่พร้อมใช้งาน: เพื่อสร้างชุดข้อมูล Scenario อุบัติเหตุที่มีโครงสร้างที่สมบูรณ์และถูกต้องตามหลักเหตุผล ซึ่งสามารถส่งออกในรูปแบบมาตรฐาน (เช่น ASAM OpenSCENARIO) เพื่อนำไปใช้งานในการทดสอบจำลอง (Simulation) ระบบขับขี่อัตโนมัติได้อย่างมีประสิทธิภาพและตรงเป้าหมาย
\end{enumerate}

\section{ประโยชน์ที่คาดว่าจะได้รับ}\label{sec:expected-benefits}

\begin{enumerate}
    \item การเพิ่มประสิทธิภาพในการค้นพบ Edge-Case: กรอบการทำงานนี้จะช่วยให้นักวิจัยและผู้พัฒนาระบบสามารถลดจำนวนเหตุการณ์จำลองที่ไม่จำเป็นลงได้อย่างมาก เนื่องจากมีการใช้ ODD เป็นไกด์ไลน์ในการกรองและควบคุมการสร้าง Scenario ให้มุ่งเน้นเฉพาะสถานการณ์ที่อยู่ในขอบเขตการทำงานของ ADS และท้าทายระบบจริง ๆ เท่านั้น ซึ่งนำไปสู่การประหยัดเวลาและทรัพยากรในการทดสอบ
    \item การยกระดับความน่าเชื่อถือของชุดทดสอบ: ชุดทดสอบที่สร้างขึ้นจาก Knowledge Graph มีความถูกต้องเชิงโครงสร้างและรักษาความต่อเนื่องเชิงเหตุผล (Causal Continuity) ของเหตุการณ์อุบัติเหตุ ทำให้ผลการประเมินระบบขับขี่อัตโนมัติมีความน่าเชื่อถือและสอดคล้องกับความเป็นจริงมากขึ้น
    \item การเป็นรากฐานสำหรับงานวิจัยต่อยอด: Knowledge Graph ที่สร้างขึ้นจากข้อมูลอุบัติเหตุจริงและผสานรวมกับเงื่อนไข ODD สามารถทำหน้าที่เป็นแหล่งข้อมูลความรู้เชิงความหมายที่มีโครงสร้าง ซึ่งเป็นประโยชน์อย่างยิ่งในการพัฒนากฎความปลอดภัย (Safety Rules), การสร้าง Ontology สำหรับการขับขี่อัตโนมัติ, หรือการพัฒนาเครื่องมือประเมินความเสี่ยงอื่น ๆ ในอนาคต
    \item การสนับสนุนการรับรองความปลอดภัยตามมาตรฐาน: กรอบการทำงานนี้ช่วยให้มั่นใจได้ว่า Scenario ที่ใช้ในการทดสอบมีความสอดคล้องกับเงื่อนไขการปฏิบัติงานที่กำหนด (ODD ของ JAMA) ซึ่งเป็นปัจจัยสำคัญในการดำเนินการและสนับสนุนกระบวนการขอการรับรองความปลอดภัยของยานยนต์อัตโนมัติ
\end{enumerate}

\section{ขอบเขต}\label{sec:scope}

\subsection{ขอบเขตของข้อมูล}\label{subsec:data-scope}
\paragraph{}รายงานจาก CIREN (Crash Injury Research and Engineering Network) ของสหรัฐอเมริกา และรายงานจาก GIDAS (German In-Depth Accident Study) ข้อมูลนำเข้าเหล่านี้อยู่ในรูปแบบของรายงานอุบัติเหตุที่เป็นข้อความแบบไม่มีโครงสร้าง (Unstructured Text Reports) ซึ่งจำเป็นต้องผ่านกระบวนการสกัดข้อมูลที่มีโครงสร้าง (Structured Data Extraction) โดยใช้ Schema-guided LLM ก่อนนำไปสร้างเป็น Knowledge Graph ทั้งนี้ ข้อจำกัดด้านการใช้งานคือ ข้อมูลทั้งหมดจะถูกใช้เพื่อสกัดเอนทิตี ความสัมพันธ์ และเงื่อนไขที่จำเป็นสำหรับการสร้าง Knowledge Graph และ Scenario จำลองเท่านั้น โดยไม่รวมถึงข้อมูลส่วนบุคคลหรือข้อมูลระบุตัวตนอื่น ๆ ที่เกี่ยวข้องกับบุคคลในรายงาน

\subsection{ขอบเขตของงาน}\label{subsec:work-scope}

\paragraph{}ขอบเขตของงานวิจัยนี้ครอบคลุมกิจกรรมหลักตั้งแต่การประมวลผลข้อมูลอุบัติเหตุไปจนถึงการสร้างชุดทดสอบที่มีโครงสร้างที่มุ่งเน้นเป้าหมาย โดยสามารถสรุปขอบเขตของการดำเนินงานได้ดังนี้:

\begin{enumerate}
    \item การพัฒนากรอบการทำงาน: พัฒนากรอบการทำงาน KGs-Augmented Testsuite Generator ซึ่งใช้ Schema-guided LLM ในการสกัดข้อมูล และใช้ Knowledge Graph (KG) ในการจัดเก็บข้อมูลเชิงความหมายพร้อมรองรับ Inference Engine
    \item การบูรณาการ ODD: นำ Operational Design Domain (ODD) ที่กำหนดโดย JAMA มาผสานรวมเข้ากับ Knowledge Graph เพื่อทำหน้าที่เป็นเงื่อนไขในการกรองและควบคุมการสร้าง Scenario ให้มุ่งเป้าหมายเฉพาะ Edge-Case ที่เกี่ยวข้องกับขอบเขตการทำงานของระบบ
    \item การสร้างผลลัพธ์: สร้างชุดข้อมูล Scenario อุบัติเหตุที่มีโครงสร้างสมบูรณ์ ซึ่งสามารถส่งออกในรูปแบบมาตรฐานของอุตสาหกรรม เช่น ASAM OpenSCENARIO ที่พร้อมนำไปใช้งานในสภาพแวดล้อมจำลอง
\end{enumerate}

\section{เครื่องมือและเทคโนโลยีที่ใช้}\label{sec:tools-and-tech}
\subsection{ฮาร์ดแวร์ที่ใช้ในการปฏิบัติงาน}\label{subsec:hardware-used}
\begin{itemize}
\item Operating System: Ubuntu 24.04.2 LTS
\item Processor: AMD Ryzen 7 5700G
\item Graphic card: Nvidia RTX 4000 Ada generation
\item Memory: 46GB
\item Storage: 1TB
\end{itemize}

\subsection{ซอฟต์แวร์ที่ใช้ในการปฏิบัติงาน}\label{subsec:software-used}
\begin{enumerate}
    \item Protégé: ใช้ในการออกแบบ Schema ของ Knowledge Graph
    \item Carla: เป็นโปรแกรมจำลองสถานการณ์การขับขี่แบบ Open
    \item MATLAB: ใช้สำหรับงานวิเคราะห์ข้อมูลโครงสร้างและเครือข่ายของถนน
    \item Large Language Models (LLMs): ใช้ GPT-4o ใช้สำหรับงานประมวลผลภาษาธรรมชาติ เพื่อช่วยในการ สร้าง (generation) และจัดการข้อมูลสำหรับสถานการณ์จำลอง และการสร้าง Knowledge Graph ตามที่ ระบุในแผนภาพระบบ
\end{enumerate}

\subsection{ภาษาโปรแกรมที่ใช้ในการพัฒนา}\label{subsec:programming-languages}
\begin{enumerate}
\item Python: ใช้สำหรับการพัฒนา Framework หลักในการสกัดข้อมูลอุบัติเหตุ การสร้าง Knowledge Graph และการผสานรวม ODD
\item Cypher: ใช้สำหรับการสืบค้นและจัดการข้อมูลในฐานข้อมูล Knowledge Graph
\item SQL: ใช้สำหรับการจัดการและสืบค้นข้อมูลในฐานข้อมูลเชิงสัมพันธ์ (Relational Database)
\item Shell Scripting: ใช้สำหรับการจัดการงานอัตโนมัติและการตั้งค่าสภาพแวดล้อมการพัฒนา
\item LaTeX: ใช้สำหรับการจัดทำรายงานและเอกสารทางวิชาการ
\item MATLAB: ใช้สำหรับการวิเคราะห์ข้อมูลโครงสร้างและเครือข่ายของถนน
\end{enumerate}

\section{แผนปฏิบัติงานสหกิจ}\label{sec:work-plan}

\begin{table}[h!]
    \centering
    \caption{ตารางสรุปแผนการดำเนินงานวิจัย}
    \label{tab:project-timeline-revised}
\begin{tabular}{|c|p{7cm}|c|c|c|c|c|}
\hline
\multicolumn{1}{|c|}{ลำดับ} & \multicolumn{1}{|c|}{กิจกรรมหลัก} & \multicolumn{1}{|c|}{พ.ค.} & \multicolumn{1}{|c|}{มิ.ย.} & \multicolumn{1}{|c|}{ก.ค.} & \multicolumn{1}{|c|}{ส.ค.} & \multicolumn{1}{|c|}{ก.ย.}\\
\hline
1 & การเรียนรู้พื้นฐานและการทบทวนงานวิจัย (LLM, KG, ODD)
  & \cellcolor{gray!40} & \cellcolor{gray!40} &  &  &  \\
\hline
2 & การสรุปแผนวิจัยโดยละเอียดและการออกแบบ Schema/Ontology สำหรับอุบัติเหตุ
  &  & \cellcolor{gray!40} & \cellcolor{gray!40} &  &  \\
\hline
3 & การจัดหาและเตรียมชุดข้อมูลอุบัติเหตุ (CIREN/GIDAS) สำหรับการสกัด
  &  & \cellcolor{gray!40} & \cellcolor{gray!40} &  &  \\
\hline
4 & การพัฒนา Schema-guided LLM สำหรับการสกัดข้อมูลอุบัติเหตุที่มีโครงสร้าง
  &  &  & \cellcolor{gray!40} & \cellcolor{gray!40} &  \\
\hline
5 & การสร้าง Knowledge Graph (KG) และการพัฒนา Inference Engine (รวม ODD)
  &  &  &  & \cellcolor{gray!40} & \cellcolor{gray!40} \\
\hline
6 & การบูรณาการระบบทั้งหมด (LLM $\rightarrow$ KG $\rightarrow$ Inference) และการสร้างชุดสถานการณ์ทดสอบเบื้องต้น
  &  &  &  & \cellcolor{gray!40} & \cellcolor{gray!40} \\
\hline
7 & การทดสอบหลัก (Main Experiment) และการสร้าง Edge-Case Scenario จำนวนมาก
  &  &  &  &  & \cellcolor{gray!40} \\
\hline
8 & การประเมินผลสถานการณ์ที่สร้างขึ้นและการวิเคราะห์สรุปผล
  &  &  &  &  & \cellcolor{gray!40} \\
\hline
9 & การเขียนรายงานฉบับสมบูรณ์และการเตรียมการเพื่อเผยแพร่ผลงานวิจัย
  &  &  &  & \cellcolor{gray!40} & \cellcolor{gray!40} \\
\hline
\end{tabular}
\end{table}
\chapter{เอกสารและงานวิจัยที่เกี่ยวข้อง}

\paragraph{}
บทนี้เป็นการทบทวนวรรณกรรม เอกสาร และงานวิจัยที่เกี่ยวข้องกับแนวคิดหลักและวิธีการที่ใช้ในการวิจัยนี้ โดยมีจุดมุ่งหมายเพื่อสร้างฐานความรู้ที่แข็งแกร่งและระบุช่องว่างทางการวิจัยที่โครงการนี้มุ่งเน้นแก้ไข เนื้อหาจะแบ่งออกเป็นแนวคิดพื้นฐานเกี่ยวกับระบบขับขี่อัตโนมัติ การสร้าง Scenario โครงสร้างข้อมูล Knowledge Graph และการใช้ LLM ในการประมวลผลภาษาธรรมชาติ

\section{แนวคิดพื้นฐานและทฤษฎีที่เกี่ยวข้อง}

\subsection{Ego Vehicle}
\paragraph{}
ในบริบทของการจำลองสถานการณ์และการทดสอบระบบขับขี่อัตโนมัติ คำว่า Ego vehicle หมายถึง ยานพาหนะหลักที่กำลังถูกทดสอบหรือประเมินผล Ego Vehicle คือรถยนต์ที่ติดตั้งระบบขับขี่อัตโนมัติ (ADS) ซึ่งเป็นหัวใจสำคัญของการทดลอง โดยมุมมอง, การรับรู้, การตัดสินใจ, และการกระทำทั้งหมดของรถคันนี้จะถูกบันทึกและวิเคราะห์เพื่อประเมินประสิทธิภาพและความปลอดภัยของระบบ

\subsection{Operational Design Domain}\label{sec:ODD}
\paragraph{}
Operational Design Domain (ODD) คือชุดของเงื่อนไขการปฏิบัติงานที่กำหนดไว้ล่วงหน้า เช่น สภาพภูมิอากาศ สภาพถนน ความเร็วสูงสุด ซึ่งระบบขับขี่อัตโนมัติ (ADS) ได้รับการออกแบบมาให้ทำงานได้อย่างปลอดภัย ODD เป็นแนวคิดที่สำคัญอย่างยิ่งในการประเมินความปลอดภัย เนื่องจากช่วยกำหนดขอบเขตการทดสอบให้ชัดเจน งานวิจัยนี้ได้ใช้ ODD ที่กำหนดโดย Japan Automobile Manufacturers Association (JAMA)เป็นข้อจำกัดหลักในการกรองและสร้าง Scenario เพื่อให้ชุดทดสอบมีความสอดคล้องกับขีดความสามารถของระบบที่กำลังประเมิน

\subsection{Knowledge Graph}\label{sec:KG}
\paragraph{}
Knowledge Graph เป็นรูปแบบการนำเสนอข้อมูลเชิงความหมาย (Semantic Data Structure) ที่ใช้โหนด (Nodes) และขอบ (Edges/Relationships) เพื่อแสดงถึงเอนทิตี (Entities) และความสัมพันธ์ระหว่างเอนทิตีเหล่านั้น บทบาทของ KG ในงานวิจัยนี้คือการทำหน้าที่เป็น Semantic Backbone สำหรับข้อมูลอุบัติเหตุ ทำให้สามารถจัดเก็บข้อมูลที่สกัดจากรายงานอุบัติเหตุในรูปแบบที่มีโครงสร้างและอนุมาน (Inference) ข้อมูลที่ขาดหายไปได้ นอกจากนี้ยังช่วยรักษาความต่อเนื่องเชิงเหตุผล (Causal Continuity) ของเหตุการณ์ที่เกิดขึ้นในอุบัติเหตุ

\subsection{การประมวลผลภาษาธรรมชาติ}\label{sec:NLP}
\paragraph{}
การประมวลผลภาษาธรรมชาติ หรือ Natural Language Processing (NLP) เป็นสาขาย่อยหนึ่งของปัญญาประดิษฐ์ (AI) และวิทยาการคอมพิวเตอร์ ที่มุ่งเน้นการสร้างปฏิสัมพันธ์ระหว่างคอมพิวเตอร์กับภาษามนุษย์ เป้าหมายหลักของ NLP คือการทำให้คอมพิวเตอร์สามารถ "เข้าใจ" ตีความ และสร้างภาษาของมนุษย์ได้ในลักษณะที่มีประโยชน์ ซึ่งครอบคลุมงานหลากหลายประเภท เช่น การสกัดข้อมูล (Information Extraction) การแปลภาษาด้วยเครื่อง (Machine Translation) และการวิเคราะห์ความรู้สึก (Sentiment Analysis) ในงานวิจัยนี้ เทคนิค NLP เป็นหัวใจสำคัญในการแปลงรายงานอุบัติเหตุที่อยู่ในรูปแบบข้อความที่ไม่มีโครงสร้าง ให้กลายเป็นข้อมูลเชิงลึกที่มีความหมายและนำไปใช้ต่อได้

\subsection{Large Language Model}
\paragraph{}
Large Language Model (LLM) เป็นโมเดลปัญญาประดิษฐ์ที่ได้รับการฝึกฝนด้วยข้อมูลจำนวนมหาศาล เพื่อทำความเข้าใจและสร้างภาษาธรรมชาติ LLM มีความสามารถในการสกัดข้อมูลที่ซับซ้อนจากข้อความที่ไม่มีโครงสร้าง (Unstructured Text) งานวิจัยนี้ใช้ LLM ในการสกัดข้อมูลอุบัติเหตุจากรายงานที่เป็นข้อความ โดยมีการนำเทคนิค Schema-guided LLM มาใช้เพื่อเพิ่มความน่าเชื่อถือและความสม่ำเสมอของข้อมูลที่สกัดได้

\subsection{Schema-guided Large Language Model}\label{sec:sgLLM}
\paragraph{}
LLM เป็นเครื่องมือที่มีประสิทธิภาพในการประมวลผลภาษาธรรมชาติ (NLP) \cite{khot2024prompting} งานวิจัยนี้ใช้เทคนิค Schema-guided LLM เพื่อควบคุมและกำหนดทิศทางการสกัดข้อมูลจากรายงานอุบัติเหตุที่เป็นข้อความ (Unstructured Text) ให้เป็นข้อมูลที่มีโครงสร้าง (Structured Data) ตาม Schema ที่ออกแบบไว้ล่วงหน้า \cite{liyan2022analysis} การควบคุมด้วย Schema นี้ช่วยเพิ่มความน่าเชื่อถือและความสม่ำเสมอของข้อมูลที่สกัดได้ ก่อนนำไปสร้างเป็น Knowledge Graph \cite{liyan2022analysis}

Schema-guided LLM ถูกใช้เพื่อแก้ไขปัญหาความไม่สมบูรณ์และความกำกวมของข้อมูลในรายงานอุบัติเหตุจริง (เช่น CIREN หรือ GIDAS) \cite{nhtsa_ciren, gidas_study} ซึ่งรายงานเหล่านี้มักถูกบันทึกในรูปแบบข้อความอิสระ (Free Text) ที่ขาดมาตรฐาน การใช้ Schema เป็น Blueprint หรือ Ontology ที่กำหนดไว้ล่วงหน้าจะทำหน้าที่เป็น "สัญญา" ในการสกัดเอนทิตี, คุณลักษณะ, และความสัมพันธ์ที่จำเป็นให้ครบถ้วน

เพื่อเพิ่มความเข้าใจในกระบวนการสกัดข้อมูลโดย Schema-guided LLM ขอนำเสนอตัวอย่างการแปลงรายงานอุบัติเหตุที่ไม่มีโครงสร้าง (Narrative) ให้อยู่ในรูปแบบข้อมูลที่มีโครงสร้าง (Extracted Feature) ซึ่งเป็นขั้นตอนสำคัญก่อนการสร้าง Knowledge Graph \cite{bagschik2018ontology}:

\begin{table}[htbp]
\centering
\caption{ตัวอย่างการสกัดข้อมูลโดย Schema-guided LLM: รายงานอุบัติเหตุ C00013}
\label{tab:sgllm_example}
\begin{tabularx}{\textwidth}{|p{0.48\textwidth}|X|}
\hline
\rowcolor{gray!20} รายงานอุบัติเหตุ (Unstructured Text: Case C00013) & ข้อมูลที่มีโครงสร้าง (Structured Data/Extracted Feature) \\
\hline
A two-vehicle collision occurred at a signalized urban intersection during daylight hours. Vehicle 1, a red fire truck traveling eastbound, entered on a green signal but came to a complete stop in the middle of the intersection. Vehicle 2, a white Honda compact SUV traveling southbound, entered against a red signal and struck the right passenger side of Vehicle 1. & \textbf{1. World/Environment:} \newline
\quad $\bullet$ Road Type: Urban intersection \newline
\quad $\bullet$ Signal Status: Functioning properly \newline
\quad $\bullet$ Time of Day: Daylight hours \newline
\quad $\bullet$ Road Condition: Dry \newline
\textbf{2. Actors:} \newline
\quad $\bullet$ Vehicle 1 (Target): Fire truck, traveling eastbound \newline
\quad $\bullet$ Vehicle 2 (Ego/ADS Candidate): White Honda compact SUV, traveling southbound \newline
\textbf{3. Scenario Sequence (Events):} \newline
\quad $\bullet$ Event 1 (V1): Stop in intersection (despite green light) \newline
\quad $\bullet$ Event 2 (V2): Encroachment (entered against red light) \newline
\quad $\bullet$ Event 3: Collision (V2 struck V1's right passenger side) \newline
\textbf{4. Outcome Metrics:} \newline
\quad $\bullet$ Injury Severity (V1 Driver): Moderate \newline
\quad $\bullet$ Injury Severity (V2 Driver): Minor \\
\hline
\end{tabularx}
\end{table}

\paragraph{}
ข้อมูลที่มีโครงสร้างดังกล่าว (ซึ่งอาจถูกเรียกว่า Extracted Feature หรือ JSON/XML output) เป็นรากฐานสำคัญที่ช่วยให้การแปลงเป็น Knowledge Graph ในขั้นตอนถัดไปเป็นไปได้อย่างแม่นยำและสม่ำเสมอ นอกจากนี้ การใช้ Schema ยังช่วยลดปัญหาความผิดพลาดในการอนุมานของ LLM (Hallucination) โดยเฉพาะในการสกัดความสัมพันธ์เชิงเหตุผลที่ซับซ้อน

\subsection{กรณีขอบเขต (Edge Case)}
\paragraph{}
หนึ่งในความท้าทายที่สำคัญที่สุดในการทดสอบระบบขับขี่อัตโนมัติ (ADS) คือการค้นหาสถานการณ์ที่เรียกว่า กรณีขอบเขต (Edge Case) ซึ่งหมายถึงสถานการณ์ที่เกิดขึ้นไม่บ่อยนัก มีความซับซ้อน หรืออยู่ ณ ขีดจำกัดของความสามารถที่ระบบได้รับการออกแบบมาให้รับมือ \cite{koopman2017autonomous} สถานการณ์เหล่านี้คือจุดที่ระบบมีโอกาสล้มเหลวหรือทำงานผิดพลาดได้มากที่สุด ตัวอย่างที่เข้าใจง่ายคือ "คนเดินถนนที่ปรากฏตัวจากมุมอับสายตาหลังรถบัสที่จอดอยู่" ซึ่งเป็นเหตุการณ์ที่คาดเดาได้ยากและต้องการการตอบสนองที่รวดเร็วจากระบบ

\paragraph{}
อย่างไรก็ตาม Edge Case ที่เกิดขึ้นในโลกแห่งความเป็นจริงมักมีความซับซ้อนกว่านั้นมาก โดยอาจประกอบด้วยลำดับเหตุการณ์ที่ต่อเนื่องกัน การตัดสินใจที่ผิดพลาดของผู้ขับขี่หลายคน หรือปัจจัยแวดล้อมที่ไม่ปกติ ดังแสดงในรูปที่~\ref{fig:edge_case_example} ซึ่งเป็นตัวอย่างสถานการณ์ที่สกัดมาจากรายงานอุบัติเหตุจริง สถานการณ์นี้ถือเป็น Edge Case ที่ดีเยี่ยมสำหรับการทดสอบ เนื่องจากเกี่ยวข้องกับการที่ยานพาหนะคันหนึ่งขับข้ามเส้นทึบเข้ามาในเลนสวนทาง ทำให้ผู้ขับขี่ทั้งสองฝ่ายต้องตัดสินใจหักหลบพร้อมกันในเวลาอันสั้น และจบลงด้วยการชนประสานงา ซึ่งเป็นสถานการณ์ที่ทดสอบความสามารถของ ADS ในการประเมินความเสี่ยงและเลือกการกระทำที่ลดความรุนแรงของอุบัติเหตุ (Damage Mitigation) ได้เป็นอย่างดี

\begin{figure}[htbp]
    \centering
    \includegraphics[width=1\textwidth]{images/edge-case-example}
    \caption{ตัวอย่าง Edge Case ที่ซับซ้อนจากรายงานอุบัติเหตุจริง (CIREN) แสดงลำดับเหตุการณ์ที่นำไปสู่การชนประสานงา}
    \label{fig:edge_case_example}
\end{figure}

\subsection{ประสิทธิภาพการค้นพบกรณีขอบเขต (Edge-Case Discovery Efficiency)}
\paragraph{}
การค้นพบ Edge Case ตามที่กล่าวมาข้างต้นมักขาดประสิทธิภาพ กล่าวคือ ต้องสิ้นเปลืองทรัพยากรและเวลาในการสร้างสถานการณ์จำลอง (Scenario) ทั่วไปที่ไม่ท้าทายระบบเป็นจำนวนมาก เพียงเพื่อจะเจอ Edge Case ที่มีความหมายเพียงไม่กี่กรณี แนวคิดเรื่อง ประสิทธิภาพการค้นพบกรณีขอบเขต (Edge-Case Discovery Efficiency) จึงเกิดขึ้นเพื่อวัดผลความสามารถของกระบวนการทดสอบ โดยนิยามว่าคือ อัตราส่วนระหว่างจำนวน Edge Case ที่ค้นพบ ต่อจำนวนสถานการณ์จำลองทั้งหมดที่ถูกสร้างขึ้น

\paragraph{}
งานวิจัยที่เกี่ยวข้องอย่าง \textit{LLMScenario} ได้แสดงให้เห็นถึงความท้าทายนี้ โดยชี้ว่าชุดข้อมูลการขับขี่จริงส่วนใหญ่มักประกอบด้วยสถานการณ์การขับขี่ปกติและปลอดภัย (Normal Safe Scenarios) แต่สถานการณ์ที่มีความเสี่ยงสูง (Extreme Risky Scenarios) ซึ่งถือเป็น Edge Case นั้นมีจำนวนน้อยมาก \cite{chang2023llmscenario} เป้าหมายหลักของงานวิจัยนี้จึงเป็นการเพิ่มประสิทธิภาพดังกล่าวให้สูงที่สุด โดยใช้ Knowledge Graph และ Operational Design Domain (ODD) เป็นกลไกสำคัญในการกรองและมุ่งเป้าการสร้างสถานการณ์จำลองไปยังขอบเขตที่ท้าทายระบบโดยตรง เพื่อลดการสร้าง Scenario ที่ไม่จำเป็นและเร่งการค้นพบ Edge Case ใหม่ให้รวดเร็วยิ่งขึ้น
\section{งานวิจัยที่เกี่ยวข้อง}

\subsection{การสร้างสถานการณ์จำลองจากรายงานอุบัติเหตุ}
\paragraph{}
มีงานวิจัยหลายฉบับที่ได้สำรวจการใช้โมเดลภาษาขนาดใหญ่ (LLM) เพื่อแปลงรายงานอุบัติเหตุเป็นสถานการณ์จำลองสำหรับทดสอบระบบขับขี่อัตโนมัติ (ADS) \cite{khot2024prompting} อย่างไรก็ตาม วิธีการเหล่านี้มักประสบปัญหาสำคัญสองประการคือ ความไม่น่าเชื่อถือของสถานการณ์จำลองที่สร้างขึ้น และการขาดกลไกที่ชัดเจนในการมุ่งเป้าไปยังกรณีขอบเขต (Edge-Case) ที่เกี่ยวข้องกับขอบเขตการทำงานของ ADS โดยตรง ซึ่งงานวิจัยฉบับนี้มุ่งเน้นการแก้ไขปัญหาดังกล่าว

\subsection{การวิเคราะห์อุบัติเหตุด้วย Knowledge Graph}{\label{sec:related_kg_analysis}}
\paragraph{}

\begin{figure}[htbp]
    \centering
    \includegraphics[width=0.9\textwidth]{images/kg-analysis-example}
    \caption{ตัวอย่าง Ontology ของ Knowledge Graph ที่ใช้ในการจำลองความสัมพันธ์ของปัจจัยต่างๆ ในอุบัติเหตุจราจร (ดัดแปลงจาก \cite{liyan2022analysis})}
    \label{fig:kg_analysis_example}
\end{figure}

มีการประยุกต์ใช้ Knowledge Graph (KG) ในการวิเคราะห์อุบัติเหตุจราจร เพื่อจัดระเบียบและแสดงความสัมพันธ์เชิงสาเหตุของปัจจัยต่างๆ ที่นำไปสู่อุบัติเหตุ แนวทางนี้ช่วยให้นักวิจัยสามารถทำความเข้าใจองค์ประกอบที่ซับซ้อนของอุบัติเหตุได้อย่างเป็นระบบ โดยมองแต่ละปัจจัยเป็นเอนทิตี (Entity) ที่เชื่อมโยงกันด้วยความสัมพันธ์ (Relationship) ดังแสดงในรูปที่~\ref{fig:kg_analysis_example} ซึ่งเป็นตัวอย่าง Ontology สำหรับอุบัติเหตุจากงานวิจัยของ Zhang และคณะ~\cite{liyan2022analysis}


\paragraph{}
จากภาพจะเห็นว่า KG สามารถเชื่อมโยงข้อมูลจากหลากหลายมิติเข้าด้วยกัน เช่น \textbf{ACCIDENT} (อุบัติเหตุ) เกิดขึ้นกับ \textbf{PEOPLE} (บุคคล) และ \textbf{VEHICLE} (ยานพาหนะ) ในสภาพ \textbf{ENVIRONMENT} (สิ่งแวดล้อม) และ \textbf{ROAD} (ถนน) ที่เฉพาะเจาะจง ซึ่งการใช้โครงสร้าง KG นี้ช่วยในการอนุมานข้อมูลที่อาจขาดหายไปและเพิ่มความเข้าใจในภาพรวมของเหตุการณ์ได้เป็นอย่างดี อย่างไรก็ตาม งานวิจัยในกลุ่มนี้มักมุ่งเน้นที่การ วิเคราะห์เพื่อความเข้าใจ มากกว่าการนำไปประยุกต์ใช้เพื่อ สร้างชุดทดสอบที่มีเป้าหมายเฉพาะ ซึ่งเป็นช่องว่างที่งานวิจัยฉบับนี้ต้องการจะเติมเต็ม

\subsection{การประยุกต์ใช้ ODD ในการทดสอบ}

\begin{figure}[htbp]
    \centering
    \includegraphics[width=1\textwidth]{images/jama-odd-example}
    \caption{ตัวอย่างการจำแนกสถานการณ์ตามกรอบการทำงาน ODD ของ JAMA \cite{jama2022framework}}
    \label{fig:jama_odd_example}
\end{figure}

\paragraph{}
มีงานวิจัยและกรอบการทำงานในอุตสาหกรรมหลายฉบับที่ได้เสนอแนวคิดในการใช้ออนโทโลยี (Ontology) หรือ Operational Design Domain (ODD) เพื่อจัดหมวดหมู่และกำหนดขอบเขตของการสร้างสถานการณ์จำลองสำหรับยานยนต์อัตโนมัติ \cite{bagschik2018ontology} แนวทางนี้ช่วยให้การทดสอบมีเป้าหมายที่ชัดเจนและเป็นระบบมากขึ้น โดยหนึ่งในกรอบการทำงานที่เป็นที่ยอมรับอย่างกว้างขวางคือ \textbf{Automated Driving Safety Evaluation Framework} โดย \textbf{JAMA} \cite{jama2022framework}

\paragraph{}
กรอบการทำงานของ JAMA ได้แบ่งประเภทของสถานการณ์การขับขี่ตามเงื่อนไขต่างๆ เช่น สภาพถนน, สภาพอากาศ, และรูปทรงของถนน เพื่อกำหนดขอบเขตการทำงานที่ปลอดภัยของระบบ ADS อย่างชัดเจน ดังแสดงในรูปที่~\ref{fig:jama_odd_example} ซึ่งจำแนกสถานการณ์ออกเป็น 3 ส่วนหลักคือ: \textbf{Preventable} (ป้องกันได้), \textbf{Unpreventable} (ป้องกันไม่ได้) ภายในขอบเขตที่คาดการณ์ได้ (Reasonably Foreseeable), และสถานการณ์ที่อยู่นอกขอบเขต ODD โดยสิ้นเชิง (Out of ODD) การจัดหมวดหมู่นี้ช่วยให้ผู้พัฒนาระบบสามารถออกแบบการทดสอบที่สอดคล้องกับความสามารถของ ADS ได้อย่างแม่นยำ


\paragraph{}
อย่างไรก็ตาม แม้แนวทางเหล่านี้จะช่วยกำหนดขอบเขตการทดสอบได้ดี แต่ยังไม่มีการผสานรวม ODD เข้ากับโครงสร้าง Knowledge Graph และ LLM อย่างเป็นระบบ เพื่อแก้ไขปัญหาประสิทธิภาพในการค้นพบ Edge-Case โดยตรง ซึ่งเป็นช่องว่างสำคัญที่งานวิจัยฉบับนี้มุ่งเน้นที่จะพัฒนา
\subsection{ความแตกต่างและช่องว่างทางการวิจัย}
\paragraph{}
หัวใจสำคัญของปัญหาที่งานวิจัยนี้มุ่งเน้นแก้ไขคือ "ประสิทธิภาพในการค้นพบกรณีขอบเขต" (Edge-Case Discovery Efficiency) ที่ต่ำในวิธีการทั่วไป งานวิจัยก่อนหน้าอย่าง \textit{LLMScenario} ได้แสดงให้เห็นอย่างชัดเจนว่าข้อมูลการขับขี่ในโลกแห่งความเป็นจริงส่วนใหญ่มหาศาลประกอบด้วยสถานการณ์ที่ปลอดภัยและเกิดขึ้นเป็นปกติ (Normal Safe Scenarios) ในขณะที่สถานการณ์ที่มีความเสี่ยงสูง (Extreme Risky Scenarios) ซึ่งเป็น Edge-Case ที่มีความสำคัญต่อการทดสอบนั้นมีอยู่น้อยมาก \cite{chang2023llmscenario} ทำให้การสร้างสถานการณ์จำลองโดยขาดการชี้นำเป้าหมายต้องสร้างเหตุการณ์ซ้ำๆ จำนวนมากจนกว่าจะพบเหตุการณ์ใหม่ที่ท้าทายระบบอย่างแท้จริง ดังแสดงในรูปที่~\ref{fig:llmscenario_distribution}

\begin{figure}[htbp]
    \centering
    \includegraphics[width=0.9\textwidth]{images/llmscenario-distribution}
    \caption{การกระจายตัวของสถานการณ์ที่แสดงให้เห็นถึงความหนาแน่นของ Normal Safe Scenarios (กลุ่มสีแดงตรงกลาง) เทียบกับ Extreme Risky Scenarios ที่กระจัดกระจาย ซึ่งชี้ให้เห็นถึงความยากในการค้นพบเหตุการณ์หายาก (ดัดแปลงจาก \cite{chang2023llmscenario})}
    \label{fig:llmscenario_distribution}
\end{figure}

\paragraph{}
งานวิจัยฉบับนี้จึงเติมเต็มช่องว่างดังกล่าว โดยนำเสนอแนวทางการแก้ปัญหาแบบบูรณาการเป็นครั้งแรก ซึ่งเป็นการสังเคราะห์จุดแข็งของเทคโนโลยีสามส่วนเข้าไว้ด้วยกัน ได้แก่ 1) ความน่าเชื่อถือของ Knowledge Graph, 2) ความสามารถของ Schema-guided LLM, และ 3) การกำหนดขอบเขตที่ชัดเจนของ Operational Design Domain (ODD) การผสานรวมเทคโนโลยีทั้งสามส่วนนี้เข้าไว้ในกรอบการทำงานเดียว (ดังแสดงในรูปที่~\ref{fig:system_architecture_overview}) ทำให้เกิดเป็นแนวทางใหม่ที่มุ่งเน้นการแก้ไขปัญหา Edge-Case Discovery Efficiency โดยเฉพาะ ซึ่งแตกต่างจากงานวิจัยก่อนหน้าที่มักจะมุ่งเน้นเพียงเทคโนโลยีใดเทคโนโลยีหนึ่งเท่านั้น

\begin{figure}[htbp]
    \centering
    \includegraphics[width=0.8\textwidth]{images/system-architecture-overview}
    \caption{แผนภาพสถาปัตยกรรมภาพรวมของกรอบการทำงาน ที่แสดงการบูรณาการ ODD, Knowledge Graph, และ LLM เข้าด้วยกันเพื่อสร้างชุดทดสอบ}
    \label{fig:system_architecture_overview}
\end{figure}

\section{เทคโนโลยีและเครื่องมือที่ใช้}

\subsection{เครื่องมือประมวลผลภาษาธรรมชาติและข้อมูล}
\begin{itemize}
    \item \textbf{Large Language Model (LLM):} งานวิจัยนี้เลือกใช้โมเดล \textbf{GPT-4o} \cite{openai2024gpt4o} เป็นเครื่องมือหลักในการทำ Schema-guided Extraction เหตุผลสำคัญที่เลือกใช้โมเดลนี้คือความสามารถแบบหลายรูปแบบ (Multimodality) โดยเฉพาะความสามารถในการประมวลผลภาพ (Vision) ซึ่งทำให้ GPT-4o สามารถวิเคราะห์ รายงานที่เป็นข้อความ (Textual Narrative) ได้พร้อมกัน ความสามารถนี้ช่วยเพิ่มความแม่นยำในการสกัดข้อมูลที่มีความสัมพันธ์เชิงพื้นที่ (Spatial Relationship) เช่น ทิศทางการเคลื่อนที่ของรถยนต์ และตำแหน่งที่เกิดการชน ซึ่งเป็นข้อมูลที่อาจกำกวมหากอ่านจากข้อความเพียงอย่างเดียว

    \item \textbf{Knowledge Graph Database:} ใช้แพลตฟอร์มฐานข้อมูลเชิงกราฟ เช่น Neo4j \cite{neo4j} ซึ่งเหมาะสมกับการจัดเก็บข้อมูลที่มีความสัมพันธ์ซับซ้อนอย่าง KG และรองรับการทำ Inference Engine เพื่อตรวจสอบข้อจำกัดของ ODD
\end{itemize}

\subsection{แหล่งข้อมูลอุบัติเหตุ}\label{subsec:accident-data-sources}
\paragraph{}
งานวิจัยนี้อาศัยข้อมูลจากฐานข้อมูลอุบัติเหตุจริงเชิงลึกที่เปิดเผยต่อสาธารณะ เพื่อใช้เป็นข้อมูลนำเข้าในการสกัดสถานการณ์จำลอง แหล่งข้อมูลหลักประกอบด้วยฐานข้อมูลสำคัญสองแห่งคือ CIREN และ GIDAS ซึ่งมีลักษณะและตัวอย่างข้อมูลดังต่อไปนี้

\subsubsection{CIREN (Crash Injury Research and Engineering Network)}
\paragraph{}
CIREN เป็นเครือข่ายวิจัยการบาดเจ็บจากอุบัติเหตุของหน่วยงาน NHTSA ในสหรัฐอเมริกา \cite{nhtsa_ciren} จุดเด่นของ CIREN คือรายงานที่มีความละเอียดสูงมาก ประกอบด้วยคำบรรยายลำดับเหตุการณ์ (Narrative) ข้อมูลเชิงวิศวกรรม และข้อมูลทางการแพทย์อย่างครบถ้วน ดังตัวอย่างกรณีศึกษาในรูปที่~\ref{fig:ciren_example}

\begin{figure}[h!]
    \centering
    \begin{minipage}{0.48\textwidth}
        \includegraphics[width=\linewidth]{images/ciren-case-example}
    \end{minipage}\hfill
    \begin{minipage}{0.48\textwidth}
        \footnotesize
        \textenglish{V1 was traveling northbound in lane five of a five lane controlled access roadway. V2 was traveling in lane four next to V1. A non-contact Medium-Heavy truck was traveling next to V2 in lane three when it began to change lanes to the left forcing V2 to also change lanes to the left. When changing lanes to the left the left side of V2's trailer contacted the right side of V1. This contact forced V1 to depart the roadway where the left side of V1 contacted a concrete barrier before continuing to travel forward across all five lanes to depart the roadway on the right, where the front of V1 contacted the guardrail face before coming to final rest.}
    \end{minipage}
    \caption{ตัวอย่างกรณีศึกษาและคำอธิบายเหตุการณ์จากฐานข้อมูล CIREN (Case \#1-10-2017-003-09)}
    \label{fig:ciren_example}
\end{figure}

\subsubsection{GIDAS (German In-Depth Accident Study)}
\paragraph{}
GIDAS เป็นโครงการรวบรวมข้อมูลอุบัติเหตุเชิงลึกในประเทศเยอรมนี \cite{gidas_study} ซึ่งมีขนาดใหญ่และครอบคลุมความรุนแรงของอุบัติเหตุที่หลากหลาย ตั้งแต่กรณีเล็กน้อยไปจนถึงรุนแรง ทำให้เหมาะสำหรับการวิเคราะห์เชิงสถิติเพื่อทำความเข้าใจการกระจายตัวของข้อมูลอุบัติเหตุจริง ดังตัวอย่างกรณีศึกษาในรูปที่~\ref{fig:gidas_example}

\begin{figure}[h!]
    \centering
    \begin{minipage}{0.48\textwidth}
        \includegraphics[width=\linewidth]{images/gidas-case-example}
    \end{minipage}\hfill
    \begin{minipage}{0.48\textwidth}
        \footnotesize
        \textenglish{Participant 01 (VW Golf) was driving on the K9013 in the direction of Oelsa. He skidded on a downhill section in a right-hand bend on a wet road surface. The vehicle understeers and collides with participant 02 (Volvo XC60), who is driving on the K9013 in the opposite direction. As a result of the collision, the car 02 slides into the embankment. Both occupants of participant 01 and the driver of participant 02 are slightly injured.}
    \end{minipage}
    \caption{ตัวอย่างกรณีศึกษาและคำอธิบายเหตุการณ์จากฐานข้อมูล GIDAS}
    \label{fig:gidas_example}
\end{figure}

\paragraph{}
ข้อมูลจากแหล่งข้อมูลทั้งสองดังที่แสดงในตัวอย่าง มีความละเอียดเพียงพอต่อการสกัดเอนทิตีและความสัมพันธ์เพื่อสร้างเป็น Knowledge Graph ได้อย่างสมบูรณ์

\subsection{มาตรฐานและสภาพแวดล้อมจำลอง}\label{subsec:ch2_standards}
\paragraph{}
ผลลัพธ์สุดท้ายของกรอบการทำงานวิจัยนี้ ถูกออกแบบให้สามารถส่งออกสถานการณ์จำลอง (Scenario) ในรูปแบบไฟล์ที่เข้ากันได้กับมาตรฐานอุตสาหกรรมยานยนต์ นั่นคือ \textbf{ASAM OpenSCENARIO} ซึ่งเป็นรูปแบบมาตรฐานที่ใช้ในการอธิบายลำดับเหตุการณ์ พฤติกรรมของ Actors, และเงื่อนไขต่างๆ ในการทดสอบระบบขับขี่อัตโนมัติ การใช้มาตรฐานนี้ช่วยให้ชุดทดสอบที่สร้างขึ้นสามารถนำไปใช้งานในสภาพแวดล้อมจำลอง (Simulation Environment) ที่หลากหลายได้ เช่น CARLA, Esmini, VTD เป็นต้น

\paragraph{}
เพื่อแสดงให้เห็นภาพผลลัพธ์ที่เป็นรูปธรรม ขอนำเสนอตัวอย่างโค้ดของไฟล์ OpenSCENARIO (`.xosc`) ซึ่งเป็นไฟล์ XML ที่อธิบายสถานการณ์จำลอง "รถยนต์ขับตัดหน้า (Cut-in)" ฉบับย่อ ดังแสดงในโค้ดตัวอย่างที่~\ref{lst:openscenario_example}

\newpage % --- คำสั่งขึ้นหน้าใหม่ ---

\begin{lstlisting}[language=XML, caption={ตัวอย่างโค้ดไฟล์ ASAM OpenSCENARIO ฉบับย่อ ที่อธิบายสถานการณ์ Cut-in}, label={lst:openscenario_example}]
<OpenSCENARIO>
  <Entities>
    <ScenarioObject name="Ego">
      <Vehicle name="DefaultVehicle" vehicleCategory="car"/>
    </ScenarioObject>
    <ScenarioObject name="Adversary">
      <Vehicle name="DefaultVehicle" vehicleCategory="car"/>
    </ScenarioObject>
  </Entities>

  <Storyboard>
    <Init>
      <Actions>
        <Private entityRef="Ego">
           </Private>
        <Private entityRef="Adversary">
           </Private>
      </Actions>
    </Init>
    <Story name="CutInStory">
      <Act name="CutInAct">
        <ManeuverGroup name="AdversaryManeuver">
          <Actors>
            <EntityRef entityRef="Adversary"/>
          </Actors>
          <Maneuver name="AdversaryCutIn">
            <Event name="AdversaryLaneChange" priority="parallel">
              <Action name="LaneChangeAction">
                <PrivateAction>
                  <LaneChangeAction>
                    <LaneChangeTarget>
                      <RelativeTargetLane entityRef="Ego" value="0"/>
                    </LaneChangeTarget>
                  </LaneChangeAction>
                </PrivateAction>
              </Action>
              <StartTrigger>
                </StartTrigger>
            </Event>
          </Maneuver>
        </ManeuverGroup>
        <StartTrigger/>
      </Act>
    </Story>
  </Storyboard>
</OpenSCENARIO>
\end{lstlisting}

\paragraph{}
จากโค้ดตัวอย่างจะเห็นองค์ประกอบสำคัญต่างๆ เช่น ส่วน `<Entities>` ใช้สำหรับประกาศ Actors ที่เกี่ยวข้อง (รถ Ego และรถคู่กรณี), ส่วน `<Init>` ใช้วางตำแหน่งเริ่มต้นของรถแต่ละคัน และส่วน `<Storyboard>` ซึ่งเป็นหัวใจหลัก ใช้อธิบายลำดับเหตุการณ์และพฤติกรรมที่จะเกิดขึ้น เช่น การเปลี่ยนเลนเพื่อขับตัดหน้า (`<LaneChangeAction>`) กรอบการทำงานวิจัยนี้จะทำหน้าที่สร้างไฟล์ที่มีโครงสร้างลักษณะนี้ขึ้นมาโดยอัตโนมัติจากข้อมูลอุบัติเหตุจริง
\chapter{ปัญหาและสมมติฐาน}\label{ch:problems-hypotheses}

\section{ปัญหาที่พบในการปฏิบัติงาน}\label{sec:problems}
\paragraph{}

ในการดำเนินงานตามกรอบการทำงาน KGs-Augmented Testsuite Generator Framework มีความท้าทายและปัญหาทางเทคนิคหลายประการที่เกิดขึ้น ซึ่งเกี่ยวข้องกับการจัดการข้อมูลที่ซับซ้อน การทำงานร่วมกันของเทคโนโลยีที่แตกต่างกัน และคุณภาพของข้อมูลนำเข้า โดยสามารถสรุปปัญหาหลักที่พบได้ดังนี้:

\subsection{ปัญหาด้านการสกัดข้อมูลและการสร้าง Knowledge Graph}\label{subsec:data-extraction-kg}

\paragraph{} ปัญหาหลักที่พบในการสกัดข้อมูลจากรายงานอุบัติเหตุและการสร้าง Knowledge Graph มีดังนี้:

\begin{enumerate}[label=\arabic*.)]
    \item ความไม่สมบูรณ์ของข้อมูลในรายงานอุบัติเหตุ: รายงานอุบัติเหตุที่มาจากแหล่งข้อมูลอย่าง CIREN หรือ GIDAS มักมีข้อมูลบางส่วนที่ขาดหายไป กำกวม หรือขัดแย้งกันเอง ซึ่งทำให้การสกัดข้อมูลโดยใช้ Schema-guided LLM มีความท้าทายอย่างมากในการรับรองความถูกต้อง (Fidelity) และความน่าเชื่อถือของเอนทิตีที่ถูกสกัด
    \item ความผิดพลาดในการอนุมานของ LLM: แม้จะใช้ Schema-guided LLM เพื่อควบคุมทิศทางในการสกัดข้อมูลแล้ว แต่โมเดลภาษาขนาดใหญ่ก็ยังคงมีแนวโน้มที่จะสร้างข้อความที่ผิดพลาดหรือข้อมูลที่ไม่สอดคล้องกับความเป็นจริง (Hallucination) โดยเฉพาะเมื่อต้องสกัดความสัมพันธ์เชิงเหตุผล (Causal Relationships) ที่ซับซ้อน
    \item ความท้าทายในการสร้าง Knowledge Graph ที่สม่ำเสมอ: การแปลงข้อมูลที่มีโครงสร้างที่ถูกสกัดมาให้อยู่ในรูปแบบ Knowledge Graph ที่สอดคล้องและสม่ำเสมอ (Consistent Schema) นั้นทำได้ยาก เนื่องจากข้อมูลอุบัติเหตุแต่ละกรณีมีความแตกต่างกันอย่างมาก ทำให้ต้องมีการปรับแก้โครงสร้างกราฟและกฎการสร้างความสัมพันธ์ (Triples) อยู่เสมอ
\end{enumerate}

\subsection{ปัญหาด้านการบูรณาการ ODD และประสิทธิภาพ}\label{subsec:odd-integration-performance}

\begin{enumerate}[label=\arabic*.)]
    \item ความซับซ้อนของการกำหนดกฎ ODD: การกำหนดกฎเชิงตรรกะที่เข้มงวดของ Operational Design Domain (ODD) ให้เป็นเงื่อนไขที่ใช้ในการสืบค้น (Inference Query) ภายใน Knowledge Graph นั้นมีความซับซ้อนสูง หากกำหนดกฎไม่ละเอียดพอ อาจทำให้เกิดการกรองที่หลวมเกินไป และยังคงสร้าง Scenario ที่ไม่เกี่ยวข้อง หรือหากกำหนดกฎที่เข้มงวดเกินไป อาจทำให้พลาด Edge-Case ที่มีความสำคัญไป
    \item ประสิทธิภาพของการประมวลผลกราฟ: เมื่อ Knowledge Graph มีขนาดใหญ่ขึ้น การเรียกใช้ Inference Engine เพื่อประเมินและค้นหา Scenario ที่ละเมิดกฎ ODD หรือสอดคล้องกับ Edge-Case จะใช้ทรัพยากรการประมวลผลและเวลาที่เพิ่มขึ้นอย่างมาก ซึ่งส่งผลต่อประสิทธิภาพโดยรวมของกรอบการทำงานในการค้นพบ Edge-Case ใหม่ ๆ
\end{enumerate}

\subsection{ปัญหาด้านผลลัพธ์และการนำไปใช้}\label{subsec:output-usage}

\paragraph{}
ความท้าทายในการส่งออกรูปแบบมาตรฐาน: การแปลงข้อมูล Scenario จากโครงสร้าง Knowledge Graph ที่มีความละเอียดสูง ไปสู่รูปแบบไฟล์มาตรฐานของอุตสาหกรรม เช่น ASAM OpenSCENARIO ต้องอาศัยการทำ Mapping ที่แม่นยำเพื่อรับประกันว่า Scenario ที่สร้างขึ้นจะสามารถรันในสภาพแวดล้อมจำลองได้อย่างถูกต้อง โดยไม่เกิดข้อผิดพลาดในการแปลความหมายของตัวแปรและเอนทิตี


\section{การวิเคราะห์ปัญหา}\label{sec:problem-analysis}
\paragraph{}

การวิเคราะห์ปัญหาที่เกิดขึ้นในการปฏิบัติงานมีความจำเป็นอย่างยิ่ง เพื่อให้สามารถระบุแนวทางแก้ไขที่เหมาะสมและปรับปรุงประสิทธิภาพของกรอบการทำงาน KGs-Augmented Testsuite Generator ได้อย่างตรงจุด การวิเคราะห์ปัญหาหลัก ๆ ที่พบมีดังนี้:

\subsection{การวิเคราะห์ปัญหาด้านข้อมูลและความน่าเชื่อถือ}\label{subsec:data-reliability-analysis}

\paragraph{}จากการวิเคราะห์ปัญหาที่กล่าวมาข้างต้น สามารถสรุปที่มาของปัญหา ได้ดังนี้

\begin{enumerate}[label=\arabic*.)]
    \item ปัญหาจากความไม่สมบูรณ์และกำกวมของข้อมูล:
    \begin{itemize}
        \item สาเหตุ: รายงานอุบัติเหตุจริง CIREN และ GIDAS ถูกบันทึกโดยมนุษย์ในรูปแบบข้อความอิสระ (Free Text) ซึ่งมีลักษณะเป็นการตีความและสังเกตการณ์ที่แตกต่างกัน ทำให้ขาดมาตรฐานในการให้ข้อมูลที่สม่ำเสมอ และมีโอกาสเกิดการละเว้นข้อมูลบางส่วนที่สำคัญต่อการสร้าง Scenario
        \item ผลกระทบ: ทำให้ Schema-guided LLM ไม่สามารถสกัดข้อมูลที่มีโครงสร้างได้อย่างสมบูรณ์ และส่งผลให้ Knowledge Graph ที่สร้างขึ้นมีช่องว่างของข้อมูล (Missing Triples) ซึ่งลดความน่าเชื่อถือและความแม่นยำของ Scenario จำลองที่ต้องอาศัยความต่อเนื่องเชิงเหตุผล
    \end{itemize}
    \item ปัญหาความผิดพลาดในการอนุมานของ LLM:
    \begin{itemize}
        \item สาเหตุ: แม้จะมีการใช้ Schema เป็นแนวทาง แต่ LLM ยังมีข้อจำกัดในการทำความเข้าใจบริบททางฟิสิกส์ (Physical Constraints) หรือกฎหมายที่ซับซ้อน ทำให้เกิดการสร้างข้อมูลที่ผิดพลาด (Hallucination) หรือการสกัดความสัมพันธ์เชิงเหตุผลที่ไม่ถูกต้อง
        \item ผลกระทบ: ทำให้ Knowledge Graph มีข้อมูลที่ผิดพลาดแฝงอยู่ ซึ่งหากนำไปสร้างเป็น Scenario จะทำให้ได้ชุดทดสอบที่ไม่มีความสมจริงหรือไม่สามารถเกิดขึ้นได้จริงในทางปฏิบัติ ส่งผลให้ประสิทธิภาพในการประเมิน ADS ลดลง
    \end{itemize}
\end{enumerate}

\subsection{การวิเคราะห์ปัญหาด้านการจัดการ Knowledge Graph และ ODD}\label{subsec:kg-odd-management-analysis}

\paragraph{}จากการวิเคราะห์ปัญหาที่กล่าวมาข้างต้น สามารถสรุปที่มาของปัญหา ได้ดังนี้

\begin{enumerate}[label=\arabic*.)]
    \item ปัญหาความซับซ้อนของการกำหนดกฎ ODD:
    \begin{itemize}
        \item สาเหตุ: การแปลงคำนิยามของ Operational Design Domain (ODD) ซึ่งเป็นแนวคิดที่ค่อนข้างเป็นนามธรรม ให้เป็นกฎเชิงตรรกะที่เข้มงวดสำหรับการสืบค้นใน Knowledge Graph (KG) นั้นต้องอาศัยความเชี่ยวชาญและการตีความที่แม่นยำ
        \item ผลกระทบ: หากการกำหนดกฎไม่แม่นยำ จะส่งผลกระทบโดยตรงต่อวัตถุประสงค์หลักของงานวิจัย กล่าวคือ การกรอง Scenario ที่ไม่เกี่ยวข้องออกไปทำได้ไม่ดีพอ (เกิด False Positives) ทำให้จำนวนเหตุการณ์ที่ต้องสร้างใหม่จนกว่าจะพบ Edge-Case ใหม่ยังคงสูงอยู่
    \end{itemize}
    \item ปัญหาประสิทธิภาพการประมวลผลกราฟขนาดใหญ่:
    \begin{itemize}
        \item สาเหตุ: การทำงานของ Inference Engine บน Knowledge Graph ที่ขยายตัวอย่างต่อเนื่อง (เมื่อมีการเพิ่มรายงานอุบัติเหตุเข้าไป) จำเป็นต้องมีการประมวลผลความสัมพันธ์จำนวนมหาศาลเพื่อหา Scenario ที่สอดคล้องกับกฎ ODD หรือละเมิดความปลอดภัย
        \item ผลกระทบ: ประสิทธิภาพการสืบค้นและเวลาในการตอบสนอง (Query Latency) ลดลงอย่างมีนัยสำคัญ เมื่อฐานข้อมูล Knowledge Graph เติบโตขึ้น ซึ่งเป็นอุปสรรคต่อการนำกรอบการทำงานนี้ไปใช้ในระดับอุตสาหกรรมที่ต้องมีการประมวลผลข้อมูลจำนวนมากแบบเรียลไทม์หรือเกือบเรียลไทม์
    \end{itemize}
\end{enumerate}

\section{สมมติฐานหรือแนวทางในการแก้ไข}\label{sec:hypotheses}
\paragraph{}

จากปัญหาที่ได้วิเคราะห์ไว้ในกระบวนการพัฒนากรณีศึกษา (Scenario) โดยใช้ KGs-Augmented Testsuite Generator Framework มีสมมติฐานและแนวทางแก้ไขหลายประการที่ถูกนำมาพิจารณาและประยุกต์ใช้เพื่อเพิ่มประสิทธิภาพ ความน่าเชื่อถือ และความสามารถในการค้นพบ Edge-Case ของระบบ

\subsection{การเสริมสร้างความน่าเชื่อถือของการสกัดข้อมูล}\label{subsec:improving-data-reliability}

\paragraph{} การแก้ไขปัญหาด้านข้อมูลและความน่าเชื่อถือของการสกัดข้อมูล มีสมมติฐานและแนวทางแก้ไขดังนี้

\begin{enumerate}[label=\arabic*.)]
    \item การใช้เทคนิค Multi-Step Prompting สำหรับ LLM: เพื่อแก้ไขปัญหาความผิดพลาดในการอนุมานของ LLM และความไม่สมบูรณ์ของข้อมูล มีการตั้งสมมติฐานว่าการแยกกระบวนการสกัดข้อมูลที่ซับซ้อนออกเป็นขั้นตอนย่อย ๆ เช่น สกัดเอนทิตี สกัดความสัมพันธ์ ตรวจสอบความสอดคล้องเชิงฟิสิกส์ จะช่วยให้ LLM มีความแม่นยำสูงขึ้นในการสกัดข้อมูลและการสร้างความสัมพันธ์เชิงเหตุผล
    \item การใช้ Cross-Validation โดยผู้เชี่ยวชาญ: กำหนดให้มีการตรวจสอบความถูกต้องของข้อมูล (Data Fidelity) ที่ถูกสกัดจาก LLM โดยผู้เชี่ยวชาญด้านการวิเคราะห์อุบัติเหตุหรือวิศวกรความปลอดภัย เพื่อปรับปรุง Schema และกฎการสกัดข้อมูลให้มีความแม่นยำสูงขึ้นก่อนนำเข้า Knowledge Graph
    \item การสร้าง Ontology สำหรับอุบัติเหตุ: พัฒนาระบบ Ontology ที่มีรายละเอียดเฉพาะสำหรับ Domain อุบัติเหตุจราจร เพื่อใช้เป็น Schema ที่เข้มงวดและเป็นมาตรฐานในการกำหนดนิยามเอนทิตีและความสัมพันธ์ใน Knowledge Graph ซึ่งจะช่วยลดปัญหาความไม่สม่ำเสมอของโครงสร้างกราฟ
\end{enumerate}

\subsection{การเพิ่มประสิทธิภาพการค้นพบ Edge-Case}\label{subsec:improving-edge-case-discovery}

\paragraph{} การแก้ไขปัญหาด้านการจัดการ Knowledge Graph และ ODD มีสมมติฐานและแนวทางแก้ไขดังนี้

\begin{enumerate}[label=\arabic*.)]
    \item การใช้กฎ ODD แบบลำดับชั้น (Hierarchical ODD Rules): เพื่อแก้ไขปัญหาความซับซ้อนในการกำหนดกฎ ODD มีสมมติฐานว่าการจัดโครงสร้างกฎ ODD ให้เป็นลำดับชั้น (เช่น เงื่อนไขทั่วไป, เงื่อนไขเฉพาะ, เงื่อนไขขอบเขต) จะช่วยให้ Inference Engine สามารถประมวลผลการกรอง Edge-Case ได้อย่างมีประสิทธิภาพมากขึ้น และลดโอกาสที่จะเกิดการกรองที่ผิดพลาด (False Filtering)
    \item การใช้ Partitioning และ Indexing ใน Knowledge Graph: เพื่อจัดการกับปัญหาประสิทธิภาพของการประมวลผลกราฟขนาดใหญ่ มีการนำเทคนิคการแบ่งส่วนข้อมูล (Partitioning) หรือการสร้างดัชนี (Indexing) เฉพาะสำหรับเอนทิตีที่เกี่ยวข้องกับ ODD และ Edge-Case เข้ามาใช้ในฐานข้อมูล Knowledge Graph เพื่อลดภาระการคำนวณของ Inference Engine ในระหว่างการสืบค้น (Query)
    \item การพัฒนา Metric ในการให้คะแนน Edge-Case (Edge-Case Scoring Metric): สร้างมาตรวัดเชิงปริมาณ (Quantitative Metric) เพื่อให้คะแนนความท้าทาย (Severity) ของ Scenario ที่สร้างขึ้น ซึ่งจะทำให้สามารถจัดลำดับความสำคัญของ Scenario ที่ถูกสร้างขึ้น และมุ่งเน้นการสร้างซ้ำเฉพาะในกลุ่มที่มีคะแนน Edge-Case สูง เพื่อให้การค้นพบเคสใหม่มีความรวดเร็วและเป็นไปตามวัตถุประสงค์ของการวิจัย
\end{enumerate}

\subsection{การปรับปรุงการส่งออก Scenario}\label{subsec:improving-scenario-export}
\paragraph{}
 การพัฒนาระบบ Mapping อัตโนมัติ: พัฒนาเครื่องมือ Mapping อัตโนมัติที่แข็งแกร่งเพื่อแปลงความสัมพันธ์และคุณสมบัติต่าง ๆ จาก Knowledge Graph ให้เป็นรูปแบบ ASAM OpenSCENARIO ที่ถูกต้องแม่นยำ โดยมีการตรวจสอบความสอดคล้องกับ Schema ของมาตรฐาน ASAM เพื่อลดข้อผิดพลาดในการแปลความหมายของ Scenario ก่อนนำไปใช้ในการจำลอง


\section{ข้อจำกัดของการศึกษา}\label{sec:limitations}
\paragraph{}

การวิจัยนี้มุ่งเน้นการพัฒนากรอบการทำงานที่เป็นแนวคิดใหม่ในการสร้างชุดทดสอบ แต่ก็มีข้อจำกัดหลายประการที่ต้องนำมาพิจารณา ซึ่งส่วนใหญ่เกี่ยวข้องกับคุณภาพของข้อมูลนำเข้า ความจำกัดของเทคโนโลยีที่ใช้ และขอบเขตการดำเนินงานที่ถูกกำหนดไว้ล่วงหน้า ดังนี้:

\subsection{ข้อจำกัดด้านข้อมูลและเทคโนโลยี}\label{subsec:data-tech-limitations}
\paragraph{} ข้อจำกัดหลักที่พบในด้านข้อมูลและเทคโนโลยี มีดังนี้
\begin{enumerate}[label=\arabic*.)]
    \item การพึ่งพาข้อมูลอุบัติเหตุในอดีต: การศึกษานี้ขึ้นอยู่กับรายงานอุบัติเหตุจริงจากฐานข้อมูลสาธารณะ เช่น CIREN และ GIDAS ซึ่งเป็นข้อมูลในอดีตและมีลักษณะที่ไม่สมบูรณ์ รวมถึงมีความเป็นอัตวิสัย (Subjectivity) ในการบันทึกของผู้รายงาน ข้อจำกัดนี้ส่งผลโดยตรงต่อคุณภาพและความแม่นยำของ Knowledge Graph ที่ถูกสร้างขึ้น
    \item ความท้าทายด้านความน่าเชื่อถือของ LLM: แม้จะมีการใช้ Schema-guided LLM เพื่อสกัดข้อมูล แต่โมเดลภาษายังคงมีแนวโน้มที่จะสร้างข้อมูลที่ผิดพลาด (Hallucination) หรือความสัมพันธ์เชิงเหตุผลที่ไม่ถูกต้อง โดยเฉพาะในสถานการณ์อุบัติเหตุที่มีความซับซ้อน ซึ่งทำให้ต้องอาศัยการตรวจสอบและปรับแก้จากผู้เชี่ยวชาญเพิ่มเติม
    \item ข้อจำกัดในการสรุปผล ODD: Operational Design Domain (ODD) ที่ใช้ในการวิจัยนี้อ้างอิงตามมาตรฐานของ JAMA เป็นหลัก ดังนั้นชุดทดสอบและผลลัพธ์ที่ได้จึงอาจไม่สามารถนำไปสรุปผลหรือนำไปประยุกต์ใช้โดยตรงกับระบบขับขี่อัตโนมัติ (ADS) ที่ถูกออกแบบมาภายใต้มาตรฐาน ODD ของผู้ผลิตหรือองค์กรอื่นที่มีนิยามที่แตกต่างกัน
    \item ปัญหาด้านการประมวลผลของ Knowledge Graph: เมื่อ Knowledge Graph เติบโตขึ้นตามจำนวนรายงานอุบัติเหตุที่เพิ่มขึ้น ประสิทธิภาพในการประมวลผลของ Inference Engine เพื่อสืบค้นและประเมินกฎ ODD จะลดลง ซึ่งอาจเป็นข้อจำกัดในการนำกรอบการทำงานนี้ไปใช้งานในระดับอุตสาหกรรมขนาดใหญ่ที่ต้องการความรวดเร็ว
\end{enumerate}

\subsection{ข้อจำกัดด้านขอบเขตการดำเนินงาน}\label{subsec:scope-limitations}
\paragraph{} ข้อจำกัดด้านขอบเขตการดำเนินงานของการวิจัยนี้ มีดังนี้
\begin{enumerate}[label=\arabic*.)]
    \item การขาดการประเมินในสภาพแวดล้อมจริง: ขอบเขตของโครงการสิ้นสุดที่การสร้างไฟล์ Scenario ที่มีโครงสร้างมาตรฐาน (เช่น ASAM OpenSCENARIO) และไม่ได้รวมถึงการดำเนินการจำลองสถานการณ์ (Simulation) หรือการทดสอบภาคสนามจริง ดังนั้น การประเมินผลกระทบที่แท้จริงของชุดทดสอบต่อประสิทธิภาพของระบบ ADS จึงอยู่นอกเหนือขอบเขตของการศึกษานี้
    \item การละเลยปัจจัยมนุษย์ในระดับละเอียด: Scenario ที่สร้างขึ้นเน้นการจับภาพเหตุการณ์ทางกายภาพและสภาพแวดล้อมเป็นหลัก แม้จะมีการเก็บข้อมูลพฤติกรรม แต่การวิเคราะห์และจำลองปัจจัยด้านมนุษย์ (Human Factors) เช่น ความผิดพลาดทางสติปัญญา หรือการตอบสนองทางอารมณ์ของผู้ขับขี่อย่างละเอียด ยังคงเป็นสิ่งที่ซับซ้อนและไม่ได้เป็นจุดเน้นหลักของกรอบการทำงานนี้
\end{enumerate}
\chapter{ขั้นตอนวิธี}\label{ch:methodology}

\section{ขั้นตอนการดำเนินงานโดยละเอียด}\label{sec:ch4_methodology}
\paragraph{}

กระบวนการดำเนินงานของโครงการวิจัยนี้เป็นไปตามกรอบการทำงาน KGs-Augmented Testsuite Generator Framework ซึ่งแบ่งออกเป็น 4 ขั้นตอนหลักที่ทำต่อเนื่องกันอย่างเป็นระบบ โดยมีเป้าหมายเพื่อเปลี่ยนรายงานอุบัติเหตุที่เป็นข้อความให้เป็น Scenario ทดสอบที่มีโครงสร้างและมุ่งเน้น Edge-Case:

\subsection{ขั้นที่ 1: การสกัดข้อมูลอุบัติเหตุที่มีโครงสร้าง (Structured Data Extraction)}\label{subsec:ch4_data_extraction}

\begin{enumerate}
    \item การเลือกแหล่งข้อมูล: กำหนดให้ใช้รายงานอุบัติเหตุเชิงลึกจากฐานข้อมูล CIREN และ GIDAS เป็นข้อมูลนำเข้า
    \item การออกแบบ Schema: ออกแบบ Schema ที่กำหนดเอนทิตี (เช่น ยานพาหนะ, ผู้ขับขี่, สภาพแวดล้อม) คุณลักษณะ (Attribute) และความสัมพันธ์ (Relationship) ที่จำเป็นต่อการสร้าง Knowledge Graph
    \item การใช้ Schema-guided LLM: ใช้โมเดลภาษาขนาดใหญ่ (LLM) ที่ถูกนำทางด้วย Schema ที่กำหนดไว้ เพื่ออ่านรายงานอุบัติเหตุ และสกัดข้อมูลสำคัญให้อยู่ในรูปแบบโครงสร้าง (Structured Data) โดยเฉพาะอย่างยิ่งการสกัดความสัมพันธ์เชิงเหตุผล (Causal Relationships)
\end{enumerate}

\subsection{ขั้นที่ 2: การสร้าง Knowledge Graph (KG Modeling)}\label{subsec:ch4_kg_modeling}

\begin{enumerate}
    \item การแปลงเป็น Triples: นำข้อมูลที่มีโครงสร้างที่ได้จากขั้นที่ 1 มาแปลงเป็น Triple Sets (Subject-Predicate-Object) เพื่อนำเข้าฐานข้อมูลกราฟ
    \item การสร้าง KG Backbone: สร้าง Knowledge Graph ซึ่งทำหน้าที่เป็น Semantic Backbone สำหรับข้อมูลอุบัติเหตุ โดยการเชื่อมโยงเอนทิตีและคุณลักษณะต่าง ๆ เข้าด้วยกัน เพื่อให้สามารถจัดเก็บข้อมูลที่ซับซ้อนได้อย่างเป็นระเบียบและรักษาความต่อเนื่องของเหตุการณ์
    \item การอนุมานข้อมูลที่ขาดหายไป: ใช้กฎการอนุมาน (Inference Rules) พื้นฐานภายใน KG เพื่อเติมเต็มช่องว่างของข้อมูลบางส่วนที่อาจขาดหายไปจากรายงานต้นฉบับ
\end{enumerate}

\subsection{ขั้นที่ 3: การบูรณาการ ODD และการค้นหา Edge-Case}\label{subsec:ch4_odd_integration}

\begin{enumerate}
    \item การกำหนดกฎ ODD: แปลงเงื่อนไข Operational Design Domain (ODD) ที่กำหนดโดย JAMA ให้เป็นกฎเชิงตรรกะที่สามารถสืบค้นได้ (Inference Query) ภายใน Knowledge Graph
    \item การผสาน ODD เข้ากับ KG: สร้างความสัมพันธ์ใหม่ใน KG เพื่อระบุสถานการณ์หรือเหตุการณ์ที่เข้าข่ายเงื่อนไข ODD ที่ท้าทาย (Edge-Case Conditions) หรือละเมิดกฎความปลอดภัย
    \item การใช้ Inference Engine: เรียกใช้ Inference Engine เพื่อสืบค้น (Query) Knowledge Graph โดยใช้กฎ ODD ที่สร้างขึ้น เพื่อกรองและเลือกเฉพาะ Scenario ที่มุ่งเป้าหมายไปยัง Edge-Case ที่เกี่ยวข้องกับขอบเขตการทำงานของระบบขับขี่อัตโนมัติ
\end{enumerate}

\subsection{ขั้นที่ 4: การสร้างไฟล์ Scenario มาตรฐาน (Standard Scenario Generation)}\label{subsec:ch4_scenario_generation}

\begin{enumerate}
    \item การทำ Mapping ข้อมูล: แปลงข้อมูล Scenario ที่ผ่านการกรอง Edge-Case แล้วจากโครงสร้าง KG ให้เข้ากับรูปแบบไฟล์มาตรฐานที่กำหนดโดยอุตสาหกรรม (เช่น ASAM OpenSCENARIO หรือ OpenDRIVE)
    \item การสร้างไฟล์ผลลัพธ์: ส่งออก Scenario ที่มีโครงสร้างสมบูรณ์และถูกต้องตามเหตุผลในรูปแบบที่พร้อมใช้งานสำหรับการจำลอง (Simulation)
\end{enumerate}

\section{การวิเคราะห์และออกแบบระบบ}\label{sec:ch4_system_design}
\paragraph{}

ระบบที่พัฒนาขึ้นมีชื่อว่า KGs-Augmented Testsuite Generator Framework ซึ่งถูกออกแบบมาในลักษณะของระบบประมวลผลข้อมูลหลายขั้นตอน (Multi-Stage Processing System) โดยมีองค์ประกอบหลักดังนี้:

\subsection{สถาปัตยกรรมระบบ}\label{subsec:ch4_architecture}

\begin{enumerate}
    \item Input Layer: รับข้อมูลนำเข้าจากรายงานอุบัติเหตุที่เป็นข้อความ (Unstructured Text Reports) จากแหล่งข้อมูล CIREN และ GIDAS
    \item Processing Layer: เป็นหัวใจของระบบ ประกอบด้วย:
    \begin{itemize}
        \item Schema-guided LLM Module: ทำหน้าที่เป็นตัวสกัดข้อมูลอัจฉริยะ โดยถูกควบคุมด้วย Ontology Schema เพื่อให้ได้ข้อมูลที่มีโครงสร้าง
        \item Knowledge Graph Database: ฐานข้อมูลเชิงกราฟที่จัดเก็บข้อมูลอุบัติเหตุในรูปแบบโหนดและความสัมพันธ์ และทำหน้าที่เป็นตัวจัดเก็บกฎ ODD และความสัมพันธ์เชิงอนุมาน
        \item Inference Engine: กลไกประมวลผลที่ใช้กฎ ODD เพื่อประเมิน Knowledge Graph และระบุ Edge-Case ที่เกี่ยวข้อง ซึ่งเป็นการลดจำนวน Scenario ที่ไม่จำเป็น
    \end{itemize}
    \item Output Layer: ส่งออก Scenario ที่ผ่านการประมวลผลแล้วในรูปแบบไฟล์มาตรฐาน (ASAM OpenSCENARIO) เพื่อเชื่อมต่อกับสภาพแวดล้อมจำลอง (Simulation Environment)
\end{enumerate}

\section{การนำไปใช้งานจริง}\label{sec:ch4_deployment}
\paragraph{}

การนำกรอบการทำงานนี้ไปใช้งานจริงมุ่งเน้นที่การสร้างชุดทดสอบที่มีคุณภาพสูงสำหรับผู้พัฒนาระบบขับขี่อัตโนมัติ:

\begin{enumerate}
    \item การติดตั้งระบบ: ระบบถูกติดตั้งในสภาพแวดล้อมการประมวลผลที่มีทรัพยากรสูง เพื่อรองรับการทำงานของ LLM และฐานข้อมูล Knowledge Graph ขนาดใหญ่
    \item การประยุกต์ใช้: วิศวกรความปลอดภัยสามารถนำเข้าชุดรายงานอุบัติเหตุใหม่ ๆ เข้าสู่ระบบ เพื่อสร้างชุด Scenario ทดสอบที่มุ่งเน้น Edge-Case ได้อย่างรวดเร็ว โดย Scenario ที่ได้จะอยู่ในรูปแบบไฟล์ OpenSCENARIO
    \item การใช้งานกับ Simulation: ไฟล์ Scenario ที่ถูกส่งออกจะถูกนำเข้าสู่แพลตฟอร์มจำลอง (เช่น Carla, Apollo) เพื่อดำเนินการทดสอบ (Run Test Cases) กับระบบขับขี่อัตโนมัติ (ADS Agent) และประเมินพฤติกรรมของระบบในสถานการณ์วิกฤตที่ตรงตามขอบเขต ODD
    \item การเผยแพร่: ผลลัพธ์และวิธีการที่ใช้ในการสร้างชุดทดสอบมีเป้าหมายเพื่อเผยแพร่ในงานประชุมวิชาการ เพื่อให้เป็นแนวทางสำหรับชุมชนนักวิจัยและอุตสาหกรรม
\end{enumerate}

\section{ปัญหาและอุปสรรคระหว่างการพัฒนา}\label{sec:ch4_issues}
\paragraph{}

ในการพัฒนาและดำเนินการโครงการ มีปัญหาและอุปสรรคหลักที่ต้องเผชิญและแก้ไข ดังนี้:

\begin{enumerate}
    \item ความไม่น่าเชื่อถือของข้อมูลเริ่มต้น: ปัญหาหลักคือความไม่สมบูรณ์และความกำกวมของข้อมูลในรายงานอุบัติเหตุ ซึ่งทำให้ Schema-guided LLM สกัดข้อมูลที่มีความผิดพลาดหรือข้อมูลขาดหายไป ซึ่งต้องแก้ไขโดยการออกแบบ Schema ให้ละเอียดและมีการตรวจสอบความถูกต้องโดยมนุษย์เพิ่มเติม
    \item ข้อจำกัดของ LLM ในการอนุมาน: แม้จะใช้ Schema เข้าช่วย แต่ LLM ยังคงสร้างความสัมพันธ์เชิงเหตุผลที่ไม่สอดคล้องกับข้อจำกัดทางฟิสิกส์ (Hallucination) ซึ่งต้องแก้ไขโดยการใช้เทคนิค Multi-Step Prompting เพื่อแยกขั้นตอนการตรวจสอบความสอดคล้องออกจากขั้นตอนการสกัดข้อมูล
    \item ประสิทธิภาพการประมวลผล Knowledge Graph ขนาดใหญ่: เมื่อจำนวน Scenario ที่ถูกสกัดและนำเข้า KG เพิ่มขึ้น ประสิทธิภาพของ Inference Engine ในการสืบค้น Edge-Case ตามกฎ ODD จะลดลงอย่างมาก แนวทางแก้ไขคือการใช้เทคนิคการทำ Indexing และ Partitioning ในฐานข้อมูลกราฟเพื่อเพิ่มความเร็วในการประมวลผล
    \item ความซับซ้อนในการทำ Mapping ผลลัพธ์: การแปลงข้อมูลจากโครงสร้าง Knowledge Graph ที่ซับซ้อนไปยังรูปแบบมาตรฐาน ASAM OpenSCENARIO ต้องอาศัยการทำ Mapping ที่แม่นยำและถูกตรวจสอบ เพื่อป้องกันข้อผิดพลาดในการแปลความหมายของ Scenario ที่นำไปใช้ในการจำลองจริง
\end{enumerate}
\chapter{ผลการศึกษา}\label{ch:results}

\paragraph{}
บทนี้จะนำเสนอผลการดำเนินงานของโครงการวิจัย KGs-Augmented Testsuite Generator Framework โดยจะแสดงให้เห็นว่ากรอบการทำงานที่พัฒนาขึ้นสามารถบรรลุวัตถุประสงค์ที่ตั้งไว้ได้อย่างไร พร้อมทั้งนำเสนอตัวอย่างผลลัพธ์ที่ได้จากระบบ การเปรียบเทียบกระบวนการก่อนและหลังการพัฒนา และรายงานผลการประเมินประสิทธิภาพของกรอบการทำงาน

\section{ผลการดำเนินงานตามวัตถุประสงค์}\label{sec:}

\paragraph{}
กรอบการทำงานที่พัฒนาขึ้นสามารถดำเนินงานได้สำเร็จและสอดคล้องกับวัตถุประสงค์หลักทั้ง 3 ข้อที่ได้ตั้งไว้ ดังนี้

\begin{enumerate}
    \item \textbf{วัตถุประสงค์ข้อที่ 1: พัฒนากรอบการสร้างชุดทดสอบที่น่าเชื่อถือ (KGs-Augmented Testsuite Generator)}
    \begin{itemize}
        \item \textbf{ผลการดำเนินงาน:} ประสบความสำเร็จในการพัฒนากรอบการทำงานที่สมบูรณ์ ซึ่งสามารถสกัดข้อมูลจากรายงานอุบัติเหตุ (Unstructured Text) ด้วย Schema-guided LLM และสร้างเป็น Knowledge Graph (KG) เพื่อใช้เป็นรากฐานเชิงความหมายของข้อมูลได้จริง ซึ่งแสดงให้เห็นถึงการบูรณาการเทคโนโลยี LLM และ KG เข้าด้วยกันอย่างเป็นระบบ
    \end{itemize}

    \item \textbf{วัตถุประสงค์ข้อที่ 2: เพิ่มประสิทธิภาพการค้นพบ Edge-Case ด้วยการบูรณาการ ODD}
    \begin{itemize}
        \item \textbf{ผลการดำเนินงาน:} บรรลุวัตถุประสงค์หลักของงานวิจัย โดยได้พัฒนากลไกการค้นหา Edge-Case ที่มีประสิทธิภาพผ่านการใช้ \textbf{ODD Modular} และเทคนิค \textbf{Query Rotation} ผลลัพธ์จากการดำเนินงานสามารถจำแนกรายงานอุบัติเหตุจริงจำนวน 318 กรณีศึกษา ออกเป็นกลุ่ม ODD Modular ที่มีความหมายได้ถึง 21 กลุ่ม ซึ่งเป็นการยืนยันว่ากรอบการทำงานสามารถกรองและมุ่งเป้าการสร้างสถานการณ์ไปยังขอบเขตที่ท้าทายระบบได้อย่างเป็นรูปธรรม ช่วยลดการสร้างเหตุการณ์ที่ไม่จำเป็นลงได้อย่างมีนัยสำคัญ
    \end{itemize}

    \item \textbf{วัตถุประสงค์ข้อที่ 3: สร้างผลลัพธ์ที่มีโครงสร้างและเป็นมาตรฐาน}
    \begin{itemize}
        \item \textbf{ผลการดำเนินงาน:} กรอบการทำงานสามารถสร้างผลลัพธ์สุดท้ายในรูปแบบของชุดพารามิเตอร์ (Parameter Set) ที่มีโครงสร้างสมบูรณ์และถูกต้องตามหลักเหตุผล ซึ่งพร้อมสำหรับนำไปใช้งานร่วมกับ Scenario Template ในมาตรฐานอุตสาหกรรมอย่าง ASAM OpenSCENARIO ได้ทันที ดังที่ได้อธิบายไว้ในขั้นตอนวิธี (บทที่ 4)
    \end{itemize}
\end{enumerate}

\section{ตัวอย่างผลลัพธ์}

\subsection{ผลการจำแนกประเภทอุบัติเหตุด้วย ODD Modular}
\paragraph{}
หัวใจสำคัญของกรอบการทำงานคือความสามารถในการจัดหมวดหมู่สถานการณ์อุบัติเหตุตามกลุ่ม ODD Modular ที่กำหนดไว้ล่วงหน้า จากการประมวลผลรายงานอุบัติเหตุทั้งหมด 318 กรณีศึกษา ระบบสามารถจำแนกและนับจำนวนเคสที่เข้าข่ายแต่ละกลุ่มได้อย่างแม่นยำ ดังแสดงในตารางที่~\ref{tab:odd_modular_results} ซึ่งแสดงให้เห็นถึงการกระจายตัวของสถานการณ์ประเภทต่างๆ

\begin{table}[htbp]
    \centering
    \caption{ผลการจำแนกประเภทอุบัติเหตุ 318 กรณีศึกษาตามกลุ่ม ODD Modular ทั้ง 21 กลุ่ม}
    \label{tab:odd_modular_distribution}
    \begin{tabular}{|l|c|c|}
        \hline
        \rowcolor{gray!20} \textbf{ODD Modular Group} & \textbf{Percentage (\%)} & \textbf{Case Count} \\
        \hline
        \multicolumn{3}{|c|}{\textbf{1. Intersections / Junctions (I) - High Complexity}} \\
        \hline
        I-1: Signalized Conflict & 3.91 & 12 \\
        I-2: Unsignalized Left-Turn & 9.09 & 29 \\
        I-3: Roundabout Conflict & 5.42 & 17 \\
        I-4: Intersection Queue Mgmt & 4.25 & 14 \\
        I-5: Crosswalk Pedestrian Conflict & 3.07 & 10 \\
        I-6: Four-Way Stop Misinterpretation & 13.70 & 44 \\
        I-7: Grade/Curvature Intersection & 5.06 & 16 \\
        \hline
        \multicolumn{3}{|c|}{\textbf{2. Traffic Maneuvers (T) - High Frequency}} \\
        \hline
        T-1: Rear-End Collision & 6.62 & 21 \\
        T-2: Aggressive Cut-In & 5.38 & 17 \\
        T-3: Unsafe Lane Change & 2.88 & 9 \\
        T-4: Following Too Closely & 2.55 & 8 \\
        T-5: Wrong-Way Driving & 7.52 & 24 \\
        T-6: Unexpected Lane Departure & 7.95 & 25 \\
        T-7: Cyclist/Motorcyclist Close Pass & 2.94 & 9 \\
        \hline
        \multicolumn{3}{|c|}{\textbf{3. Adverse Environment / Structure (E) - High Severity}} \\
        \hline
        E-1: Heavy Rainfall / Low Friction & 4.00 & 13 \\
        E-2: Nighttime / Poor Illumination & 2.74 & 9 \\
        E-3: Sun Glare / Low Sun Angle & 3.49 & 11 \\
        E-4: Construction Zone Conflict & 2.28 & 7 \\
        E-5: Fog / Dust Storm & 3.33 & 11 \\
        E-6: Road Obstacle / Debris & 1.69 & 5 \\
        E-7: Toll/Entrance Gate Merge & 2.13 & 7 \\
        \hline
    \end{tabular}
\end{table}

\subsection{กรณีศึกษา C00013: Intersection Signalized Conflict}
\paragraph{}
เพื่อแสดงให้เห็นภาพการทำงานตั้งแต่ต้นจนจบ ขอนำเสนอกรณีศึกษา C00013 ซึ่งเป็นอุบัติเหตุ ณ ทางแยกที่มีสัญญาณไฟ กรอบการทำงานได้สกัดข้อมูลจากรายงานที่เป็นข้อความ และสร้างเป็นชุดพารามิเตอร์ที่มีโครงสร้างดังรูปที่~\ref{fig:c00013_features} ซึ่งพารามิเตอร์เหล่านี้ถูกนำไปใช้สร้างสถานการณ์จำลองที่สมจริงได้ในท้ายที่สุด ดังแสดงในรูปที่~\ref{fig:c00013_simulation}

\begin{figure}[htbp]
    \centering
    \includegraphics[width=0.9\textwidth]{images/c00013_extracted_features}
    \caption{ตัวอย่างชุดพารามิเตอร์ที่มีโครงสร้างซึ่งถูกสกัดจากรายงานอุบัติเหตุ C00013}
    \label{fig:c00013_features}
\end{figure}

\begin{figure}[htbp]
    \centering
    \includegraphics[width=0.49\textwidth]{images/c00013_sim_front}
    \includegraphics[width=0.49\textwidth]{images/c00013_sim_front_col}
    \caption{ภาพจากสถานการณ์จำลองของกรณีศึกษา C00013 ที่สร้างขึ้นจากพารามิเตอร์ของระบบ (ซ้าย: มุมมองด้านหน้า, ขวา: มุมมองด้านหน้าขณะชน)}
    \label{fig:c00013_simulation}
\end{figure}

\section{การเปรียบเทียบก่อนและหลังการพัฒนา}
\paragraph{}
การนำกรอบการทำงาน KGs-Augmented Testsuite Generator มาใช้ได้เปลี่ยนแปลงกระบวนการสร้างชุดทดสอบจากการพยายามแบบไร้ทิศทางไปสู่กระบวนการที่เป็นระบบและมุ่งเน้นเป้าหมายอย่างชัดเจน ดังนี้

\begin{itemize}
    \item \textbf{ก่อนการพัฒนา:} การใช้ LLM โดยตรงเพื่อสร้างสถานการณ์จากรายงานอุบัติเหตุมีลักษณะเป็นการ "สุ่ม" ซึ่งต้องสร้างสถานการณ์ที่ไม่ท้าทายระบบ (Normal Safe Scenarios) จำนวนมากเพื่อที่จะพบ Edge-Case ที่มีความหมายเพียงไม่กี่กรณี ทำให้มีประสิทธิภาพการค้นพบกรณีขอบเขต (Edge-Case Discovery Efficiency) ที่ต่ำ และสิ้นเปลืองทรัพยากรอย่างมาก

    \item \textbf{หลังการพัฒนา:} กรอบการทำงานใหม่ใช้ KG เป็นฐานความรู้และใช้ ODD Modular เป็นตัวกรอง ทำให้กระบวนการสร้างสถานการณ์มีเป้าหมายที่ชัดเจน (Goal-Oriented) โดยมุ่งเน้นไปที่การค้นหาเคสที่ตรงตามเงื่อนไขความเสี่ยงที่กำหนดไว้เท่านั้น ส่งผลให้ ประสิทธิภาพการค้นพบกรณีขอบเขตสูงขึ้นอย่างมีนัยสำคัญ และลดจำนวนครั้งในการสร้างเหตุการณ์ซ้ำซ้อนลงได้
\end{itemize}

\section{การประเมินผล}
\paragraph{}
เพื่อประเมินคุณภาพและความสมจริงของผลลัพธ์ ได้มีการใช้แนวคิดการประเมินเชิงเทคนิคที่เรียกว่า Scenario Completeness Score หรือ R-score (Realism Score) ซึ่งเป็นการวัดความคล้ายคลึงกันระหว่าง Scene Graph ของสถานการณ์ในอุบัติเหตุจริง กับ Scene Graph ที่สร้างขึ้นจากพารามิเตอร์ของกรอบการทำงาน

\paragraph{}
ผลการประเมินเบื้องต้นพบว่าค่า R-score มีความแตกต่างกันไปในแต่ละ ODD Modular ซึ่งชี้ให้เห็นว่ากรอบการทำงานมีความสามารถในการสร้างสถานการณ์ที่มีความซับซ้อนแตกต่างกันได้ดี ดังแสดงในตารางที่~\ref{tab:r_score_results}

\begin{table}[htbp]
    \centering
    \caption{ตัวอย่างผลการประเมิน R-score ในกลุ่ม ODD Modular ต่างๆ}
    \label{tab:r_score_results}
    \begin{tabular}{|l|c|}
        \hline
        \rowcolor{gray!20} \textbf{ODD Modular Group} & \textbf{Average R-score} \\
        \hline
        T-1: Rear-End Collision & \textbf{0.9577} (สูง) \\
        \hline
        E-6: Road Obstacle - Debris & \textbf{0.6305} (ต่ำ) \\
        \hline
    \end{tabular}
\end{table}

\paragraph{}
จากตาราง ผลการประเมินชี้ว่าสถานการณ์ที่มีปฏิสัมพันธ์ระหว่างยานพาหนะที่ไม่ซับซ้อน เช่น "การชนท้าย" (Rear-End Collision) สามารถสร้างได้อย่างสมจริงและได้ R-score ที่สูง ในขณะที่สถานการณ์ที่เกี่ยวข้องกับวัตถุบนถนนแบบสุ่ม เช่น "สิ่งกีดขวาง" (Road Obstacle) ยังคงมีความท้าทายในการสกัดข้อมูลและสร้างให้สมจริง ซึ่งเป็นแนวทางสำหรับการพัฒนาต่อไปในอนาคต
\chapter{สรุปผลการศึกษาและวิจารณ์ผลการศึกษา}

\section{สรุปผลการดำเนินงาน}

สรุปผลสำเร็จของโครงการอย่างกระชับและครอบคลุม

\section{ข้อสังเกตและข้อวิจารณ์}

สะท้อนถึงประเด็นที่น่าสนใจหรือจุดที่ยังควรปรับปรุง

\section{ข้อเสนอแนะในการพัฒนาต่อไป}

เสนอแนวทางหรือฟีเจอร์เพิ่มเติมที่ควรมีในอนาคต

\section{ประสบการณ์จากการเข้าร่วมโครงการสหกิจศึกษา}

ถ่ายทอดสิ่งที่ผู้เขียนได้เรียนรู้จากการทำงานจริงในองค์กร

\bibliographystyle{plain}
\bibliography{references}

\end{document}